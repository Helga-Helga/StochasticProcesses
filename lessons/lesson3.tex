\addcontentsline{toc}{chapter}{Занятие 5. Винеровский процесс}
\chapter*{Занятие 5. Винеровский процесс}

\addcontentsline{toc}{section}{Контрольные вопросы и задания}
\section*{Контрольные вопросы и задания}

\subsubsection*{Приведите определение процесса Пуассона.}

$\left\{ N \left( t \right), \, t \geq 0 \right\} $ --- процесс Пуассона, если
\begin{enumerate}
  \item $N \left( 0 \right) = 0$;
  \item при $t_1 < t_2 < \dotsc < t_n$ события
  $$N \left( t_1 \right), N \left( t_2 \right) - N \left( t_1 \right), \dotsc,
    N \left( t_n \right) - N \left( t_{n - 1} \right) $$
  --- независимые;
  \item число событий на интервале зависит только от длины интервала,
  то есть есть однородность приращений
  $$N \left( t + s \right) - N \left( t \right) \overset{d}{=}
    N \left( s \right) \sim
    Pois \left( \lambda s \right).$$
\end{enumerate}

\subsubsection*{Запишите конечномерные распределения процесса Пуассона.}

Одномерные распределения
$$P \left\{ N \left( t \right) = k \right\} =
  e^{-\lambda t} \cdot \frac{ \left( \lambda t \right)^k}{k!}.$$

Двумерные распределения: $t_1 < t_2$.
Перейдём к приращениям
$$P \left\{ N \left( t_1 \right) = k_1, \, N \left( t_2 \right) = k_2 \right\} =
  P \left\{
    N \left( t_1 \right) = k_1, \, N \left( t_2 \right) - N \left( t_1 \right) = k_2 - k_1
  \right\} =$$
Случайная величина $N \left( t_1 \right) \sim Pois \left( \lambda t_1 \right) $, а
$N \left( t_2 \right) - N \left( t_1 \right) \sim
  Pois \left( \lambda \left( t_2 - t_1 \right) \right) $.
Совместная вероятность --- это произведение вероятностей
$$= e^{-\lambda t_1} \cdot \frac{ \left( \lambda t_1 \right)^{k_1}}{k_1!} \cdot
  e^{-\lambda \left( t_2 - t_1 \right) } \cdot
  \frac{ \left( \lambda \left( t_2 - t_1 \right) \right)^{k_2 - k_1}}{ \left( k_2 - k_1 \right)!}.$$

\subsubsection*{Какой вид имеют траектории процесса Пуассона?}

Траектория изображена на рисунке \ref{fig:3}.

\begin{figure}[h!]
  \centering
  \includegraphics[width=.4\textwidth]{./pictures/3.png}
  \caption{График пуассоновского процесса}
  \label{fig:3}
\end{figure}

\subsubsection*{Какое содержание имеет параметр процесса Пуассона?}

$N \left( t \right) $ --- число событий, произошедших до момента времени $t$.

\subsubsection*{3.3}

\textit{Задание.}
Пусть $ \left\{ N \left( t \right), \, t \geq 0 \right\} $ ---
процесс Пуассона с интенсивностью $ \lambda $.
Вычислите условное математическое ожидание
$M \left[ N \left( s \right) \; \middle| \; N \left( t \right) \right] $ для
$$0 \leq s \leq t.$$

\textit{Решение.}
Что ж такое условное математическое ожидание?

$N \left( s \right) $ и $N \left( t \right) $ --- дискретные величины, то есть
$$M \left[ N \left( s \right) \; \middle| \; N \left( t \right) \right] =
  \sum \limits_{l = 0}^{ \infty }
    l \cdot P \left\{ N \left( s \right) = l \; \middle| \; N \left( t \right) = k \right\}.$$
Отдельно посчитаем условную вероятность, а потом по ней возьмём математическое ожидание
$$P \left\{ N \left( s \right) = l \; \middle| \; N \left( t \right) = k \right\} =
  \frac{P \left\{ N \left( t \right) - N \left( s \right) = k - l \right\} P \left\{ N \left( s \right) = l \right\} }{P \left\{ N \left( t \right) = k \right\} } =$$
подставляем пуассоновские вероятности
$$= \frac{ \frac{ \left[ \lambda \left( t - s \right) \right]^{k - l}}{ \left( k - l \right)!} \cdot e^{-\lambda \left( t - s \right) }}{ \frac{ \left( \lambda t \right)^k}{k!} \cdot e^{-\lambda t}} \cdot
  \cdot \frac{ \left( \lambda s \right)^l}{l!} \cdot e^{-\lambda l} =
  \frac{k! \left( t - s \right)^{k - l} s^l}{ \left( k - l \right)! l! t^k} =
  C_k^l \cdot \left( \frac{t - s}{t} \right)^{k - l} \cdot \left( \frac{s}{t} \right)^l.$$
Имеем биномиальное распределение с параметрами $k$ и $ \frac{s}{t}$.

Вывод: при условии $N \left( t \right) = k$ мы нашли распределение
$$N \left( s \right) \sim
  B \left( k, \frac{s}{t} \right).$$

Условное математическое ожидание
$$M \left[ N \left( s \right) \; \middle| N \left( t \right) = k \right] =
  \frac{ks}{t}.$$

Ответ: $ \frac{N \left( t \right) \cdot s}{t}$.
Куда пропала сумма?

$$M \left[ N \left( s \right) \; \middle| \; N \left( t \right) = k \right] =
  \sum \limits_l
    l \cdot P \left[ N \left( s \right) = l \; \middle| \; N \left( t \right) = k \right] =$$
Нашли эту вероятность
$$= \sum \limits_l l \cdot P \left\{ Bin \left( k, \frac{s}{t} \right) = l \right\} =
MBin \left( k, \frac{s}{t} \right) =
k \cdot \frac{s}{t}.$$

Условное математическое ожидание --- это математическое ожидание по условному распределению.

\subsubsection*{3.4}

\textit{Задание.}
Пусть $N = \left\{ N \left( t \right), \, t \geq 0 \right\} $ ---
процесс Пуассона с интенсивностью $ \lambda $.
Найдите вероятность того, что первый прыжок процесса $N$ произошёл до момента времени
$s \in \left[ 0, t \right] $ при условии,
что на отрезке $ \left[ 0, t \right] $ произошло ровно $n$ прыжков.

\textit{Решение.}
Нужно найти вероятность $P$(первый прыжок произошёл до момента
$\left. s \in \left[ 0, t \right] \right| N \left( t \right) = n$).
Нужно это условие переписать через пуассоновский процесс.
Получаем
$P$(первый прыжок произошёл до момента
  $\left. s \in \left[ 0, t \right] \right| N \left( t \right) = n$) =
  $P \left\{ N \left( s \right) \geq 1 \; \middle| \; N \left( t \right) = n \right\} $.
Значения зависимые
$$P \left\{ N \left( s \right) \geq 1 \; \middle| \; N \left( t \right) = n \right\} =
  1 - P \left\{ N \left( s \right) = 0 \; \middle| \; N \left( t \right) = n \right\} =$$
Условное распределение биномиальное
\begin{gather*}
  = 1 - P \left\{ Bin \left( n, \frac{s}{t} \right) = 0 \right\} =
  1 - C_n^0 \cdot \left( \frac{s}{t} \right)^0 \cdot \left( 1 - \frac{s}{t} \right)^{n - 0} =
  1 - \left( 1 - \frac{s}{t} \right)^n.
\end{gather*}

\subsubsection*{3.5}

\textit{Задание.}
Пусть $N = \left\{ N \left( t \right), \, t \geq 0 \right\} $
является процессом Пуассона с параметром $ \lambda $.
Докажите, что при условии, что $N$ имеет ровно 1 прыжок на отрезке $ \left[ a, b \right] $,
момент этого прыжка является равномерно распределённой на отрезке $ \left[ a, b \right] $
случайной величиной.

\textit{Решение.}
Обозначим $ \tau $ --- момент прыжка на отрезке $ \left[ a, b \right] $.
Нужно найти
$P \left\{ \tau \leq t \; \middle| \; N \left( b \right) - N \left( a \right) = 1 \right\} =
  P \left\{
    N \left( t \right) - N \left( a \right) = 1 \; \middle| \;
    N \left( b \right) - N \left( a \right) = 1 \right\} $.

Изобразим процесс на графике \ref{fig:35}.

\begin{figure}[h!]
  \centering
  \includegraphics[width=.4\textwidth]{./pictures/3_5.png}
  \caption{График пуассоновского процесса}
  \label{fig:35}
\end{figure}

У пуассоновского процесса есть однородность приращений
\begin{gather*}
  P \left\{
    N \left( t \right) - N \left( a \right) = 1 \; \middle| \;
    N \left( b \right) - N \left( a \right) = 1 \right\} =
  P \left\{ N \left( t - a \right) = 1 \; \middle| \; N \left( b - a \right) = 1 \right\} = \\
  = P \left\{ Bin \left( 1, \frac{t - a}{b - a} \right) = 1 \right\} =
\end{gather*}
Это бернуллиевская величина
$$= \frac{t - a}{b - a}.$$
Получилось равномерное распределение, что и требовалось доказать.
