\addcontentsline{toc}{chapter}{Занятие 9. Линейные преобразования случайных процессов.}
\chapter*{Занятие 9. Линейные преобразования случайных процессов.}

\addcontentsline{toc}{section}{Контрольные вопросы и задания}
\section*{Контрольные вопросы и задания}

\subsubsection*{Приведите определение стационарного в широком смысле процесса.}

$ \xi \left( t \right), \, t \in T$ называется стационарным в широком смысле, если
\begin{enumerate}
  \item $m \left( t  \right) \equiv m$ (const);
  \item $K \left( t, s \right) = k \left( t + r, s + r \right), \qquad \forall t, s, r \in T$.
  Это означает, что ковариационная функция~---~это сейчас функция разности аргументов,
  то есть $K \left( t, s \right) = k \left( t - s \right) $.
\end{enumerate}

\subsubsection*{Сформулируйте теорему Бохнера про спектральное изображение ковариационной функции
                стационарного в широком смысле случайного процесса.}

Пусть $ \xi \left( t \right), \, t \in \mathbb{R}$~---~это
стационарный и непрерывный в среднем квадратическом случайный процесс.
Тогда существует конечная мера $ \mu $ на $ \mathbb{R}$ такая, что
\begin{equation*}
  k \left( t \right) =
  \int \limits_{-\infty }^{+\infty } e^{it \lambda } \mu \left( d \lambda \right),
\end{equation*}
мера $ \mu $ определяется единственным образом.

\subsubsection*{Запишите, как изменяются ковариационная функция и спектральная функция стационарного
                в широком смысле случайного процесса при применении к нему линейного
                дифференциального оператора, интегрального оператора.}

\begin{equation*}
  P \left( \frac{d}{dt} \right) \xi \left( t \right) =
  Q \left( \frac{d}{dt} \right) \eta \left( t \right),
\end{equation*}
где $ \xi $ и $ \eta $ --- гладкие в среднем квадратическом стационарные процессы.

Спектральная мера для
\begin{equation*}
  P \left( \frac{d}{dt} \right) \xi
\end{equation*}
имеет вид
\begin{equation*}
  \rho \left( d \lambda \right) =
  \left| P \left( i \lambda \right) \right|^2 \mu \left( d \lambda \right).
\end{equation*}
Следовательно,
\begin{equation*}
  \mu_{ \xi } \left( d \lambda \right) =
  \frac{ \left| Q \left( i \lambda \right) \right|^2}{ \left| P \left( i \lambda \right) \right|^2} \cdot
  \mu_{ \eta } \left( d \lambda \right).
\end{equation*}

\addcontentsline{toc}{section}{Аудиторные задачи}
\section*{Аудиторные задачи}

\subsubsection*{9.2}

\textit{Задание.}
Пусть $ \left\{ \xi \left( t \right), \, t \in \mathbb{R} \right\} $~---~стационарный
в широком смысле процесс.
Положим
\begin{equation*}
  \eta \left( t \right) =
  \sum \limits_{k = 1}^n c_k \xi \left( t + \delta_k \right), \,
  t \in \mathbb{R},
\end{equation*}
где $c_1, \dotsc, c_n, \delta_1, \dotsc, \delta_n$~---~некоторые постоянные.
Докажите, что процесс $ \left\{ \eta \left( t \right), \, t \in \mathbb{R} \right\} $
являвтся стационарным в широком смысле.
Выразите ковариационную и спектральную функции процесса $ \eta $
через ковариационную и спектральную функцию процесса $ \xi $.

\textit{Решение.}
Если $ \xi $~---~стационарный, это значит, что $M \xi \left( t \right) = m_{ \xi } = const$ и
$cov \left[ \xi \left( t \right), \xi \left( s \right) \right] =
  k_{ \xi } \left( t - s \right) $.

Проверим, что $ \eta $~---~стационарный
\begin{equation*}
  M \eta \left( t \right) =
  M \sum \limits_{k = 1}^n c_k \xi \left( t + \delta_k \right) =
\end{equation*}
Выносим сумму и коэффициенты
\begin{equation*}
  = \sum \limits_{k = 1}^n c_k M \xi \left( t + \delta_k \right) =
  m_{ \xi } \sum \limits_{k = 1}^n c_k.
\end{equation*}
Значит, математическое ожидание $ \eta $ будет тоже постоянным.

Теперь найдём ковариационную функцию для $ \eta $.
Вынесем две суммы и коэффициенты
\begin{equation*}
  cov \left[ \eta \left( t \right), \eta \left( s \right) \right] =
  \sum \limits_{k = 1}^n
    \sum \limits_{i = 1}^n
      c_k c_i cov \left[ \xi \left( t + \delta_k \right), \xi \left( s + \delta_i \right) \right] =
\end{equation*}
Такая ковариация~---~это $k_{ \xi }$ от разности аргументов
\begin{equation*}
  = \sum \limits_{k = 1}^n
    \sum \limits_{i = 1}^n c_k c_i k_{ \xi } \left( t + \delta_k - s - \delta_i \right)=
  \sum \limits_{k = 1}^n
    \sum \limits_{i = 1}^n c_k c_i k_{ \xi } \left( t - s + \delta_k - \delta_i \right).
\end{equation*}

Ответ зависит только от разности $t$ и $s$, значит, это стационарный процесс
\begin{equation*}
  k_{ \xi } \left( t \right) =
  \int \limits_{ \mathbb{R}} e^{i \lambda t} F_{ \xi } \left( d \lambda \right),
\end{equation*}
где $F_{ \xi } \left( d \lambda \right) $~---~спектральная функция для $ \xi $.

Для процесса $ \eta $ нужно найти представление
\begin{equation*}
  k_{ \eta } \left( t \right) =
  \int \limits_{ \mathbb{R}} e^{i \lambda t} F_{ \eta } \left( d \lambda \right),
\end{equation*}
где $F_{ \eta } \left( d \lambda \right) $~---~искомая функция.

Выпишем
\begin{equation*}
  k_{ \eta } \left( t \right) =
  \sum \limits_{k, j} c_k c_j k_{ \xi } \left( t + \delta_k - \delta_j \right).
\end{equation*}

Для $k_{ \xi }$ есть интегральное выражение, подставим его и попытаемся вынести интеграл за сумму
\begin{equation*}
  k_{ \eta } \left( t \right) =
  \sum \limits_{k, j = 1}^n
    c_k c_j \int \limits_{ \mathbb{R}}
      e^{i \lambda \left( t + \delta_k - \delta_j \right) } F_{ \xi } \left( d \lambda \right) =
\end{equation*}
Приведём интеграл к нужному виду.
Нужно интеграл вынести за сумму, и чтобы в интеграле получилась экспонента без всяких $ \delta $.
Получим
\begin{equation*}
  = \int \limits_{ \mathbb{R}}
      \sum \limits_{k, j = 1}^n
        c_k c_j e^{i \lambda t} e^{i \lambda \left( \delta_k - \delta_j \right) }
    F_{ \xi } \left( d \lambda \right) =
  \int \limits_{ \mathbb{R}} e^{i \lambda t} F_{ \eta } \left( d \lambda \right).
\end{equation*}

Получилось, что $F_{ \eta } \left( d \lambda \right) $~---~спектральная функция для $ \eta $, равная
\begin{equation*}
  F_{ \eta } \left( d \lambda \right) =
  \sum \limits_{k, j = 1}^n
    c_k c_j e^{i \lambda \left( \delta_k - \delta_j \right) } F_{ \xi } \left( d \lambda \right).
\end{equation*}

Плотность должна всегда быть неотрицательной.

Как понять, что такая двойная сумма неотрицательна?

Экспоненту запишем как произведение
\begin{equation*}
  \sum \limits_{k, j = 1}^n c_k c_j e^{i \lambda \left( \delta_k - \delta_j \right) } =
  \sum \limits_{k = 1}^n
    \sum \limits_{j = 1}^n c_k e^{i \lambda \delta_k} c_j e^{-i \lambda \delta_j} =
\end{equation*}
Запишем через произведение двух сумм
\begin{equation*}
  = \left( \sum \limits_{k = 1}^n c_k e^{i \lambda \delta_k} \right) \cdot
  \left( \sum \limits_{k = 1}^n c_k e^{-i \lambda \delta_k} \right) =
\end{equation*}
Это комплексно сопряжённые числа
\begin{equation*}
  = \left| \sum \limits_{k = 1}^n c_k e^{i \lambda \delta_k} \right|^2.
\end{equation*}
Получили неотрицательную величину, которая может быть плотностью.
