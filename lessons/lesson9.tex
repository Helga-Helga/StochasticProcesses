\addcontentsline{toc}{chapter}{Занятие 9. Линейные преобразования случайных процессов.}
\chapter*{Занятие 9. Линейные преобразования случайных процессов.}

\addcontentsline{toc}{section}{Контрольные вопросы и задания}
\section*{Контрольные вопросы и задания}

\subsubsection*{Приведите определение стационарного в широком смысле процесса.}

$ \xi \left( t \right), \, t \in T$ называется стационарным в широком смысле, если
\begin{enumerate}
  \item $m \left( t  \right) \equiv m$ (const);
  \item $K \left( t, s \right) = k \left( t + r, s + r \right), \qquad \forall t, s, r \in T$.
  Это означает, что ковариационная функция~---~это сейчас функция разности аргументов,
  то есть $K \left( t, s \right) = k \left( t - s \right) $.
\end{enumerate}

\subsubsection*{Сформулируйте теорему Бохнера про спектральное изображение ковариационной функции
                стационарного в широком смысле случайного процесса.}

Пусть $ \xi \left( t \right), \, t \in \mathbb{R}$~---~это
стационарный и непрерывный в среднем квадратическом случайный процесс.
Тогда существует конечная мера $ \mu $ на $ \mathbb{R}$ такая, что
\begin{equation*}
  k \left( t \right) =
  \int \limits_{-\infty }^{+\infty } e^{it \lambda } \mu \left( d \lambda \right),
\end{equation*}
мера $ \mu $ определяется единственным образом.

\subsubsection*{Запишите, как изменяются ковариационная функция и спектральная функция стационарного
                в широком смысле случайного процесса при применении к нему линейного
                дифференциального оператора, интегрального оператора.}

\begin{equation*}
  P \left( \frac{d}{dt} \right) \xi \left( t \right) =
  Q \left( \frac{d}{dt} \right) \eta \left( t \right),
\end{equation*}
где $ \xi $ и $ \eta $ --- гладкие в среднем квадратическом стационарные процессы.

Спектральная мера для
\begin{equation*}
  P \left( \frac{d}{dt} \right) \xi
\end{equation*}
имеет вид
\begin{equation*}
  \rho \left( d \lambda \right) =
  \left| P \left( i \lambda \right) \right|^2 \mu \left( d \lambda \right).
\end{equation*}
Следовательно,
\begin{equation*}
  \mu_{ \xi } \left( d \lambda \right) =
  \frac{ \left| Q \left( i \lambda \right) \right|^2}{ \left| P \left( i \lambda \right) \right|^2} \cdot
  \mu_{ \eta } \left( d \lambda \right).
\end{equation*}
