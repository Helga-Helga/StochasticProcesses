\addcontentsline{toc}{chapter}{Занятие 9. Линейные преобразования случайных процессов.}
\chapter*{Занятие 9. Линейные преобразования случайных процессов.}

\addcontentsline{toc}{section}{Контрольные вопросы и задания}
\section*{Контрольные вопросы и задания}

\subsubsection*{Приведите определение стационарного в широком смысле процесса.}

$ \xi \left( t \right), \, t \in T$ называется стационарным в широком смысле, если
\begin{enumerate}
  \item $m \left( t  \right) \equiv m$ (const);
  \item $K \left( t, s \right) = k \left( t + r, s + r \right), \qquad \forall t, s, r \in T$.
  Это означает, что ковариационная функция~---~это сейчас функция разности аргументов,
  то есть $K \left( t, s \right) = k \left( t - s \right) $.
\end{enumerate}

\subsubsection*{Сформулируйте теорему Бохнера про спектральное изображение ковариационной функции
                стационарного в широком смысле случайного процесса.}

Пусть $ \xi \left( t \right), \, t \in \mathbb{R}$~---~это
стационарный и непрерывный в среднем квадратическом случайный процесс.
Тогда существует конечная мера $ \mu $ на $ \mathbb{R}$ такая, что
\begin{equation*}
  k \left( t \right) =
  \int \limits_{-\infty }^{+\infty } e^{it \lambda } \mu \left( d \lambda \right),
\end{equation*}
мера $ \mu $ определяется единственным образом.

\subsubsection*{Запишите, как изменяются ковариационная функция и спектральная функция стационарного
                в широком смысле случайного процесса при применении к нему линейного
                дифференциального оператора, интегрального оператора.}

\begin{equation*}
  P \left( \frac{d}{dt} \right) \xi \left( t \right) =
  Q \left( \frac{d}{dt} \right) \eta \left( t \right),
\end{equation*}
где $ \xi $ и $ \eta $ --- гладкие в среднем квадратическом стационарные процессы.

Спектральная мера для
\begin{equation*}
  P \left( \frac{d}{dt} \right) \xi
\end{equation*}
имеет вид
\begin{equation*}
  \rho \left( d \lambda \right) =
  \left| P \left( i \lambda \right) \right|^2 \mu \left( d \lambda \right).
\end{equation*}
Следовательно,
\begin{equation*}
  \mu_{ \xi } \left( d \lambda \right) =
  \frac{ \left| Q \left( i \lambda \right) \right|^2}{ \left| P \left( i \lambda \right) \right|^2} \cdot
  \mu_{ \eta } \left( d \lambda \right).
\end{equation*}

\addcontentsline{toc}{section}{Аудиторные задачи}
\section*{Аудиторные задачи}

\subsubsection*{9.2}

\textit{Задание.}
Пусть $ \left\{ \xi \left( t \right), \, t \in \mathbb{R} \right\} $~---~стационарный
в широком смысле процесс.
Положим
\begin{equation*}
  \eta \left( t \right) =
  \sum \limits_{k = 1}^n c_k \xi \left( t + \delta_k \right), \,
  t \in \mathbb{R},
\end{equation*}
где $c_1, \dotsc, c_n, \delta_1, \dotsc, \delta_n$~---~некоторые постоянные.
Докажите, что процесс $ \left\{ \eta \left( t \right), \, t \in \mathbb{R} \right\} $
являвтся стационарным в широком смысле.
Выразите ковариационную и спектральную функции процесса $ \eta $
через ковариационную и спектральную функцию процесса $ \xi $.

\textit{Решение.}
Если $ \xi $~---~стационарный, это значит, что $M \xi \left( t \right) = m_{ \xi } = const$ и
$cov \left[ \xi \left( t \right), \xi \left( s \right) \right] =
  k_{ \xi } \left( t - s \right) $.

Проверим, что $ \eta $~---~стационарный
\begin{equation*}
  M \eta \left( t \right) =
  M \sum \limits_{k = 1}^n c_k \xi \left( t + \delta_k \right) =
\end{equation*}
Выносим сумму и коэффициенты
\begin{equation*}
  = \sum \limits_{k = 1}^n c_k M \xi \left( t + \delta_k \right) =
  m_{ \xi } \sum \limits_{k = 1}^n c_k.
\end{equation*}
Значит, математическое ожидание $ \eta $ будет тоже постоянным.

Теперь найдём ковариационную функцию для $ \eta $.
Вынесем две суммы и коэффициенты
\begin{equation*}
  cov \left[ \eta \left( t \right), \eta \left( s \right) \right] =
  \sum \limits_{k = 1}^n
    \sum \limits_{i = 1}^n
      c_k c_i cov \left[ \xi \left( t + \delta_k \right), \xi \left( s + \delta_i \right) \right] =
\end{equation*}
Такая ковариация~---~это $k_{ \xi }$ от разности аргументов
\begin{equation*}
  = \sum \limits_{k = 1}^n
    \sum \limits_{i = 1}^n c_k c_i k_{ \xi } \left( t + \delta_k - s - \delta_i \right)=
  \sum \limits_{k = 1}^n
    \sum \limits_{i = 1}^n c_k c_i k_{ \xi } \left( t - s + \delta_k - \delta_i \right).
\end{equation*}

Ответ зависит только от разности $t$ и $s$, значит, это стационарный процесс
\begin{equation*}
  k_{ \xi } \left( t \right) =
  \int \limits_{ \mathbb{R}} e^{i \lambda t} F_{ \xi } \left( d \lambda \right),
\end{equation*}
где $F_{ \xi } \left( d \lambda \right) $~---~спектральная функция для $ \xi $.

Для процесса $ \eta $ нужно найти представление
\begin{equation*}
  k_{ \eta } \left( t \right) =
  \int \limits_{ \mathbb{R}} e^{i \lambda t} F_{ \eta } \left( d \lambda \right),
\end{equation*}
где $F_{ \eta } \left( d \lambda \right) $~---~искомая функция.

Выпишем
\begin{equation*}
  k_{ \eta } \left( t \right) =
  \sum \limits_{k, j} c_k c_j k_{ \xi } \left( t + \delta_k - \delta_j \right).
\end{equation*}

Для $k_{ \xi }$ есть интегральное выражение, подставим его и попытаемся вынести интеграл за сумму
\begin{equation*}
  k_{ \eta } \left( t \right) =
  \sum \limits_{k, j = 1}^n
    c_k c_j \int \limits_{ \mathbb{R}}
      e^{i \lambda \left( t + \delta_k - \delta_j \right) } F_{ \xi } \left( d \lambda \right) =
\end{equation*}
Приведём интеграл к нужному виду.
Нужно интеграл вынести за сумму, и чтобы в интеграле получилась экспонента без всяких $ \delta $.
Получим
\begin{equation*}
  = \int \limits_{ \mathbb{R}}
      \sum \limits_{k, j = 1}^n
        c_k c_j e^{i \lambda t} e^{i \lambda \left( \delta_k - \delta_j \right) }
    F_{ \xi } \left( d \lambda \right) =
  \int \limits_{ \mathbb{R}} e^{i \lambda t} F_{ \eta } \left( d \lambda \right).
\end{equation*}

Получилось, что $F_{ \eta } \left( d \lambda \right) $~---~спектральная функция для $ \eta $, равная
\begin{equation*}
  F_{ \eta } \left( d \lambda \right) =
  \sum \limits_{k, j = 1}^n
    c_k c_j e^{i \lambda \left( \delta_k - \delta_j \right) } F_{ \xi } \left( d \lambda \right).
\end{equation*}

Плотность должна всегда быть неотрицательной.

Как понять, что такая двойная сумма неотрицательна?

Экспоненту запишем как произведение
\begin{equation*}
  \sum \limits_{k, j = 1}^n c_k c_j e^{i \lambda \left( \delta_k - \delta_j \right) } =
  \sum \limits_{k = 1}^n
    \sum \limits_{j = 1}^n c_k e^{i \lambda \delta_k} c_j e^{-i \lambda \delta_j} =
\end{equation*}
Запишем через произведение двух сумм
\begin{equation*}
  = \left( \sum \limits_{k = 1}^n c_k e^{i \lambda \delta_k} \right) \cdot
  \left( \sum \limits_{k = 1}^n c_k e^{-i \lambda \delta_k} \right) =
\end{equation*}
Это комплексно сопряжённые числа
\begin{equation*}
  = \left| \sum \limits_{k = 1}^n c_k e^{i \lambda \delta_k} \right|^2.
\end{equation*}
Получили неотрицательную величину, которая может быть плотностью.

\subsubsection*{9.3}

\textit{Задание.}
Пусть $ \left\{ \xi \left( t \right), \, t \in \mathbb{R} \right\} $~---~стационарный в широком
смысле процесс, непрерывный в среднем квадратическом;
$f$~---~непрерывная функция с компактным носителем.
Положим
\begin{equation*}
  \eta \left( t \right) =
  \int \limits_{-\infty }^{ \infty } f \left( t - s \right) \xi \left( s \right) ds.
\end{equation*}
Выразите спектральную функцию $F_{ \eta }$ через спектральную функцию $F_{ \xi }$.

\textit{Решение.}
Проверим, что $ \eta $~---~тоже стационарный.
Начнём с математического ожидания
\begin{equation*}
  M \eta \left( t \right) =
  M \int \limits_{-\infty }^{+\infty } f \left( t - s \right) \xi \left( s \right) ds =
\end{equation*}
Математическое ожидание и интеграл можно поменять местами
\begin{equation*}
  = \int \limits_{-\infty }^{+\infty } f \left( t - s \right) M \xi \left( s \right) ds =
  \int \limits_{-\infty }^{+\infty } f \left( t - s \right) \cdot const \cdot ds =
  \int \limits_{-\infty }^{+\infty } f \left( t - s \right) \cdot m_{ \xi } ds =
\end{equation*}
Сделаем замену переменных: $t - s = x$~---~новая переменная
\begin{equation*}
  = m_{ \xi } \int \limits_{-\infty }^{+\infty } f \left( x \right) dx
\end{equation*}
не зависит от $t$.

Значит, математическое ожидание постоянное.
Теперь ковариационная функци
\begin{equation*}
  cov \left[ \eta \left( t \right), \eta \left( s \right) \right] =
  cov \left[
    \int \limits_{-\infty }^{+\infty } f \left( t - u \right) \xi \left( u \right) du,
    \int \limits_{-\infty }^{+\infty } f \left( s - v \right) \xi \left( v \right) dv \right] =
\end{equation*}
Интегралы выносим, $f$ выносим,
остаётся под интегралами $cov \left[ \xi \left( u \right), \xi \left( v \right) \right] $, то есть
\begin{equation*}
  \int \limits_{-\infty }^{+\infty }
    \int \limits_{-\infty }^{+\infty }
      f \left( t - u \right) f \left( s - v \right) \cdot k_{ \xi } \left( u - v \right) dudv =
\end{equation*}
Сделаем две замены: $t - u = x, \, s - v = y$.
Получим
\begin{equation*}
  = \int \limits_{-\infty }^{+\infty }
    \int \limits_{-\infty }^{+\infty }
      f \left( x \right) f \left( y \right) \cdot
      k_{ \xi } \left( t - x + y - s \right)
    dx
  dy.
\end{equation*}

Интеграл зависит от разности $t - s$.
Спектральную функцию для $ \eta $ выразим через спектральную функцию для $ \xi $.
Нашли, что
\begin{gather*}
  k_{ \eta } \left( t \right) =
  \iint \limits_{ \mathbb{R}^2}
    f \left( x \right) f \left( y \right) k_{ \xi } \left( t - x + y \right) dxdy = \\
  = \iint \limits_{ \mathbb{R}^2}
    f \left( x \right) f \left( y \right)
    \int \limits_{ \mathbb{R}}
      e^{i \lambda \left( t - x + y \right) } F_{ \xi } \left( d \lambda \right) dxdy = \\
  = \int \limits_{ \mathbb{R}}
    e^{i \lambda t} \iint \limits_{ \mathbb{R}^2}
      f \left( x \right) f \left( y \right) e^{i \lambda \left( y - x \right) }
    dxdy F_{ \xi } \left( d \lambda \right).
\end{gather*}
Двойной интеграл~---~спектральная функция для $ \eta $.

Ответ получился следующий.
Спектральная функция для $ \eta $ равна
\begin{equation*}
  k_{ \eta } \left( \lambda \right) =
  \left|
    \int \limits_{-\infty }^{+\infty }
      f \left( s \right) e^{i \lambda s} ds \right|^2 F_{ \xi } \left( d \lambda \right).
\end{equation*}

При этом процесс $ \eta $ задавался как
\begin{equation*}
  \eta \left( t \right) =
  \int \limits_{-\infty }^{+\infty } f \left( t - s \right) \xi \left( s \right) ds =
  \int \limits_{-\infty }^{+\infty } f \left( s \right) \xi \left( t + s \right) ds.
\end{equation*}

\subsubsection*{9.4}

\textit{Задание.}
Найдите ковариационную функцию процессов $ \eta, \zeta $
и совместную ковариационную функцию $ \eta, \zeta $, если
\begin{equation*}
  \eta \left( t \right) =
  \xi \left( t \right) + \xi' \left( t \right) + \xi'' \left( t \right); \,
  \zeta \left( t \right) =
  3 \xi'' \left( t \right) -2 \xi \left( t \right);
\end{equation*}
а $ \xi $~---~стационарный в широком смысле процесс с ковариационной функцией
$K \left( t \right) =
  e^{-\frac{t^2}{2}}$.

\textit{Решение.}
\begin{equation*}
  K_{ \eta } \left( t, s \right) =
  cov \left[ \eta \left( t \right), \eta \left( s \right) \right] =
\end{equation*}

Напишем отдельно
\begin{gather*}
  cov \left[ \xi \left( t \right), \xi \left( s \right) \right] =
  k \left( t - s \right), \,
  cov \left[ \xi \left( t \right), \xi' \left( s \right) \right] =
  -k' \left( t - s \right), \, \\
  cov \left[ \xi' \left( t \right), \xi \left( s \right) \right] =
  k' \left( t - s \right), \,
  cov \left[ \xi' \left( t \right), \xi' \left( s \right) \right] =
  -k'' \left( t - s \right).
\end{gather*}
Ещё нужно
\begin{gather*}
  cov \left[ \xi \left( t \right), \xi'' \left( s \right) \right] =
  k'' \left( t - s \right), \,
  cov \left[ \xi'' \left( t \right), \xi \left( s \right) \right] =
  k'' \left( t - s \right), \, \\
  cov \left[ \xi' \left( t \right), \xi'' \left( s \right) \right] =
  k''' \left( t - s \right), \,
  cov \left[ \xi'' \left( t \right), \xi' \left( s \right) \right] =
  -k''' \left( t - s \right), \, \\
  cov \left[ \xi'' \left( t \right), \xi'' \left( s \right) \right] =
  k^{ \left( 4 \right) } \left( t - s \right).
\end{gather*}
Подставим в ковариацию выражение для процесса
\begin{equation*}
  = cov \left[
    \xi \left( t \right) + \xi' \left( t \right) + \xi'' \left( t \right),
    \xi \left( s \right) + \xi' \left( s \right) + \xi'' \left( s \right)
  \right] =
\end{equation*}
Слагаемые с нечётными производными сокращаются
\begin{equation*}
  = k \left( t - s \right) + k'' \left( t - s \right) +
  k^{ \left( 4 \right) } \left( t - s \right) =
\end{equation*}
Посчитаем производные
\begin{gather*}
  k' \left( t \right) =
  \left( e^{-\frac{t^2}{2}} \right) =
  -\frac{1}{2} \cdot 2te^{-\frac{t^2}{2}} =
  -te^{-\frac{t^2}{2}} =
  -tk \left( t \right), \,
  k'' \left( t \right) = -k \left( t \right) + t^2 k \left( t \right), \, \\
  k''' \left( t \right) =
  tk \left( t \right) + 2tk \left( t \right) + t^2 k \left( t \right).
\end{gather*}
Подставим производные в ковариацию
\begin{gather*}
  = k \left( t - s \right) - k \left( t - s \right) +
  \left( t - s \right)^2 k \left( t - s \right) + 3k \left( t - s \right) -
  6 \left( t - s \right)^2 k \left( t - s \right) + \\
  + \left( t - s \right)^4 k \left( t - s \right) =
  3k \left( t - s \right) - 5 \left( t - s \right)^2 k \left( t - s \right) +
  \left( t - s \right)^4 k \left( t - s \right) = \\
  = k \left( t - s \right)
  \left[ 3 - 5 \left( t - s \right)^2 + \left( t - s \right)^4 \right] =
  e^{-\frac{ \left( t - s \right)^2}{2}}
  \left[ 3 - 5 \left( t - s \right)^2 + \left( t - s \right)^4 \right] = \\
  = k_{ \eta } \left( t - s \right).
\end{gather*}

Для другого процесса~---~аналогично.

Видно, что ковариационная функция меняется сложно.

Как меняется спектральная плотность при действии дифференцирования?
Оператор, действующий на $ \xi $~---~это
\begin{equation*}
  \eta \left( t \right) =
  \xi \left( t \right) + \xi' \left( t \right) + \xi'' \left( t \right) =
  P \left( \frac{d}{dt} \right) \xi, \,
  P \left( x \right) = 1 + x + x^2.
\end{equation*}

Тогда спектральная плотность
$f_{ \eta } \left( \lambda \right) =
  \left| P \left( i \lambda \right) \right|^2
  f_{ \xi } \left( \lambda \right) $.

Спектральная плотность меняется гораздо проще чем ковариационная функция.

\subsubsection*{9.5}

\textit{Задание.}
Найдите спектральную плотность процесса
\begin{equation*}
  \eta \left( t \right) =
  \xi \left( t \right) + \xi' \left( t \right) + \xi'' \left( t \right),
\end{equation*}
если спектральная плотность стационарного процесса $ \xi $ равна
\begin{equation*}
  f_{ \xi } \left( \lambda \right) =
  \mathbbm{1}_{ \left[ 0, 1 \right] } \left( \lambda \right).
\end{equation*}

\textit{Решение.}
\begin{equation*}
  \eta \left( t \right) =
  P \left( \frac{d}{dt} \right) \xi,
\end{equation*}
где $P \left( x \right) = 1 + x + x^2$.

Тогда спектральная плотность считается по формуле
\begin{gather*}
  f_{ \eta } \left( \lambda \right) =
  \left| P \left( i \lambda \right) \right|^2 f_{ \xi } \left( \lambda \right) =
  \left| 1 + i \lambda - \lambda^2 \right|^2 \cdot
  \mathbbm{1}_{ \left[ 0, 1 \right] } \left( \lambda \right) = \\
  = \left[ \left( 1 - \lambda^2 \right)^2 + \lambda^2 \right] \cdot
  \mathbbm{1}_{ \left[ 0, 1 \right] } \left( \lambda \right) =
\end{gather*}
Раскроем скобки
\begin{equation*}
  = \left( 1 - \lambda^2 + \lambda^4 \right) \cdot
  \mathbbm{1}_{ \left[ 0, 1 \right] } \left( \lambda \right).
\end{equation*}

\subsubsection*{9.6}

\textit{Задание.}
Найдите спектральную плотность стационарного решения уравнения
$ \xi' \left( t \right) =
  \eta \left( t \right) $,
если спектральная плотность стационарного процесса $ \eta $ равна
$f_{ \eta } \left( \lambda \right) =
  \lambda^2 \mathbbm{1}_{ \left[ 1, 2 \right] } \left( \lambda \right) $.

\textit{Решение.}
Перепишем равенство в дифференциальном виде, то есть через многочлен
\begin{equation*}
  P \left( \frac{d}{dt} \right) \xi \left( t \right) =
  \eta \left( t \right).
\end{equation*}

Тогда $P \left( x \right) = x$.
Теперь $f_{ \eta }$ будет записываться как
$f_{ \eta } \left( \lambda \right) =
  \left| P \left( i \lambda \right) \right|^2
  f_{ \xi } \left( \lambda \right) $.
Теперь отсюда можно найти
\begin{equation*}
  f_{ \xi } \left( \lambda \right) =
  \frac{f_{ \eta } \left( \lambda \right) }{ \left| P \left( i \lambda \right) \right|^2} =
  \frac{f_{ \eta } \left( \lambda \right) }{ \left| i \lambda \right|^2} =
  \frac{f_{ \eta } \left( \lambda \right) }{ \lambda^2} =
\end{equation*}
Подставим $f_{ \eta }$.
Получим
\begin{equation*}
  = \frac{ \lambda^2 \mathbbm{1}_{ \left[ 1, 2 \right] } \left( \lambda \right) }{ \lambda^2 } =
  \mathbbm{1}_{ \left[ 1, 2 \right] } \left( \lambda \right).
\end{equation*}

\subsubsection*{9.7}

\textit{Задание.}
Найдите спектральную плотность стационарного решения уравнения
$ \xi'' \left( t \right) - 2 \xi' \left( t \right) =
  \eta' \left( t \right) - \eta \left( t \right) $,
если спектральная плотность стационарного процесса $ \eta $ равна
$f_{ \eta } \left( \lambda \right) =
  \lambda^2 \mathbbm{1}_{ \left[ 1, 2 \right] } \left( \lambda \right) $.

\textit{Решение.}
Перепишем равенство в виде
\begin{equation*}
  P \left( \frac{d}{dt} \right) \xi \left( t \right) =
  Q \left( \frac{d}{dt} \right) \eta \left( t \right),
\end{equation*}
где $P \left( x \right) = -2x + x^2, \, Q \left( x \right) = -1 + x$.

Тогда
\begin{equation*}
  f_{ \xi } \left( \lambda \right) =
  \frac{ \left| Q \left( i \lambda \right) \right|^2}{ \left| Q \left( i \lambda \right) \right|^2} \cdot
  f_{ \eta } \left( \lambda \right) =
  \left( \frac{ \left| 1 - 2i \lambda - \lambda^2 \right|^2}{ \left| -1 + i \lambda \right|^2} \right)^{-1} \cdot
  f_{ \eta } \left( \lambda \right) =
  \frac{1 + \lambda^2}{4 \lambda^2 + \lambda^4} \cdot f_{ \eta } \left( \lambda \right) =
\end{equation*}
Подставим плотность
\begin{equation*}
  = \frac{1 + \lambda^2}{4 \lambda^2 + \lambda^4} \cdot \lambda^2 \cdot
  \mathbbm{1}_{ \left[ 1, 2 \right] } \left( \lambda \right) =
  \frac{1 + \lambda^2}{4 + \lambda^2} \cdot
  \mathbbm{1}_{ \left[ 1, 2 \right] } \left( \lambda \right).
\end{equation*}

\addcontentsline{toc}{section}{Домашнее задание}
\section*{Домашнее задание}

\subsubsection*{9.10}

\textit{Задание.}
Пусть $ \left\{ \xi \left( t \right), \, t \in T \right\} $~---~действительнозначный
стационарный в широком смысле процесс с математическим ожиданием $m$ и спектральной плотностью
$f \left( \lambda \right) $.
Положим
$ \eta \left( t \right) = \xi \left( t \right) \cos \left( \Lambda t + \varphi \right), \,
  t \in T$,
где
\begin{equation*}
  \Lambda =
  const,
\end{equation*}
а $\varphi$~---~независимая от $ \xi $ случайна величина,
равномерно распределённая на $ \left[ 0, 2 \pi \right) $.
Докажите, что случайный процесс $ \left\{ \eta \left( t \right), \, t \in T \right\} $
является стационарным в широком смысле и найдите его спектральную функцию.

\textit{Решение.}
Если $ \xi $~---~стационарный, это значит,
что $M \xi \left( t \right) = m_{ \xi } = const$ и
$cov \left[ \xi \left( t \right), \xi \left( s \right) \right] =
  k_{ \xi } \left( t - s \right) $.

Проверим, что $ \eta $~---~стационарный
\begin{equation*}
  M \eta \left( t \right) =
  M \left[ \xi \left( t \right) \cos \left( \Lambda t + \varphi \right) \right] =
  M \xi \left( t \right) \cdot M \cos \left( \Lambda t + \varphi \right) =
\end{equation*}
Запишем математическое ожидание $ \varphi $ через интеграл от плотности
\begin{equation*}
  = m_{ \xi } \cdot \int \limits_0^{2 \pi } \cos \left( \Lambda t + x \right) dx =
\end{equation*}
Сделаем замену
\begin{equation*}
  \Lambda t + x = u \Rightarrow du = dx, \,
  x = 0 \Rightarrow u = \Lambda t, \,
  x = 2 \pi \Rightarrow u = \Lambda t + 2 \pi.
\end{equation*}
Подставив замену в интеграл, получим
\begin{equation*}
  = m_{ \xi } \cdot \int \limits_{ \Lambda t}^{ \Lambda t + 2 \pi } \cos u du =
  m_{ \xi } \cdot \left. \sin u \right|_{ \Lambda t}^{ \Lambda t + 2 \pi } =
  m_{ \xi } \left[ \sin \left( \Lambda t + 2 \pi \right) - \sin \left( \Lambda t \right) \right] =
\end{equation*}
Запишем разность синусов через произведение синуса на косинус
\begin{equation*}
  = m_{ \xi } \cdot 2 \sin \frac{ \Lambda t + 2 \pi - \Lambda t}{2} \cdot
  \cos \frac{ \Lambda t + 2 \pi + \Lambda t}{2} =
  2m_{ \xi } \cdot \sin \pi \cos \left( \Lambda t + \pi \right) =
  0.
\end{equation*}

Значит, математическое ожидание $ \eta $ будет тоже постоянным.

Теперь найдём ковариационную функцию для $ \eta $.
Получим
\begin{equation*}
  cov \left[ \eta \left( t \right), \eta \left( s \right) \right] =
  M \left[ \eta \left( t \right) \eta \left( s \right) \right] -
  M \eta \left( t \right) \cdot M \eta \left( s \right) =
\end{equation*}
Подставим выражение для случайного процесса
\begin{equation*}
  = M \left[
    \xi \left( t \right) \cos \left( \Lambda t + \varphi \right)
    \xi \left( s \right) \cos \left( \lambda s + \varphi \right) \right] =
\end{equation*}
Распишем произведение косинусов через их сумму
\begin{equation*}
  = M \left[
    \xi \left( t \right) \xi \left( s \right) \cdot \frac{1}{2} \cdot
    \cos \left( \Lambda t + \varphi - \Lambda s - \varphi \right) +
    \cos \left( \Lambda t + \varphi + \Lambda s + \varphi \right) \right] =
\end{equation*}
Упростим аргументы косинусов
\begin{equation*}
  = M \left\{
    \xi \left( t \right) \xi \left( s \right) \cdot
    \frac{ \cos \left[ \Lambda \left( t - s \right) \right] + \cos \left[ \Lambda \left( t + s \right) + 2 \varphi \right] }{2} \right\} =
\end{equation*}
Разобьём на 2 математических ожидания
\begin{gather*}
  = M \left\{
    \xi \left( t \right) \xi \left( s \right) \cdot
    \frac{ \cos \left[ \Lambda \left( t - s \right) \right] }{2} \right\} +
  M \left\{
    \xi \left( t \right) \xi \left( s \right)
    \cos \left[ \Lambda \left( t + s \right) + 2 \varphi \right] \right\} = \\
  = M \left[ \xi \left( t \right) \xi \left( s \right) \right] \cdot
  \frac{ \cos \left[ \Lambda \left( t - s \right) \right] }{2} = \\
  = \left\{
    cov \left[ \xi \left( t \right), \xi \left( s \right) \right] -
    M \xi \left( t \right) M \xi \left( s \right)
  \right\} \cdot \frac{ \cos \left[ \Lambda \left( t - s \right) \right] }{2} = \\
  = \frac{k_{ \xi } \left( t - s \right) - m_{ \xi }^2}{2} \cdot
  \cos \left[ \Lambda \left( t - s \right) \right].
\end{gather*}

Ответ зависит только от разности $t$ и $s$, значит, это стационарный процесс
\begin{equation*}
  k_{ \xi } \left( t \right) =
  \int \limits_{ \mathbb{R}} e^{i \lambda t} dF_{ \xi } \left( d \lambda \right) =
  \int \limits_{ \mathbb{R}} e^{i \lambda t} f \left( \lambda \right) d \lambda,
\end{equation*}
где $F_{ \xi } \left( d \lambda \right) $~---~спектральная функция для $ \xi $.

Для процесса $ \eta $ нужно найти представление
\begin{equation*}
  k_{ \eta } \left( t \right) =
  \int \limits_{ \mathbb{R}} e^{i \lambda t} F_{ \eta } \left( d \lambda \right),
\end{equation*}
где $F_{ \eta } \left( d \lambda \right) $~---~искомая функция.

Выпишем
\begin{equation*}
  k_{ \eta } \left( t \right) =
  \frac{k_{ \xi } \left( t \right) - m_{ \xi }^2}{2} \cdot \cos \left( \Lambda t \right).
\end{equation*}
Для $k_{ \xi }$ есть интегральное выражение, подставим его
\begin{equation*}
  k_{ \eta } \left( t \right) =
  \left[
    \frac{1}{2} \int \limits_{ \mathbb{R}} e^{i \lambda t} f \left( \lambda \right) d \lambda -
    \frac{m_{ \xi }^2}{2} \right] \cdot \cos \left( \Lambda t \right) =
\end{equation*}
Можем внести косинус и константы под знак интеграла, потому что они не зависят от аргумента,
по которому идёт интегрирование,
\begin{equation*}
  = \int \limits_{ \mathbb{R}}
    e^{i \lambda t} \cdot
    \left[ \frac{1}{2} \cdot f \left( \lambda \right) - \frac{m_{ \xi }^2}{2} \right] \cdot
    \cos \left( \Lambda t \right) d \lambda =
  \int \limits_{ \mathbb{R}} e^{i \lambda t} F_{ \eta } \left( d \lambda \right).
\end{equation*}

Получили, что $F_{ \eta } \left( d \lambda \right) $~---~спектральная функция для $ \eta $, равная
\begin{equation*}
  F_{ \eta } \left( d \lambda \right) =
  \left[ \frac{1}{2} \cdot f \left( \lambda \right) - \frac{m_{ \xi }^2}{2} \right] \cdot
  \cos \left( \Lambda t \right).
\end{equation*}
