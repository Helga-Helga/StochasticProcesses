\addcontentsline{toc}{chapter}{Занятие 7. $L_2$ теория}
\chapter*{Занятие 7. $L_2$ теория}

\addcontentsline{toc}{section}{Контрольные вопросы и задания}
\section*{Контрольные вопросы и задания}

\subsubsection*{Приведите опредедение непрерывного в среднем квадратическом случайного процесса.}

$ \xi $ непрерывен в среднем квадратическом в точке $t_0$, если
$$ \xi \left( t \right) \overset{L_2}{ \to }
  \xi \left( t_0 \right) $$
при $t \to t_0$,
то есть $M \left[ \xi \left( t \right) - \xi \left( t_0 \right) \right]^2 \to 0, \, t \to t_0$.

Случайный процесс $ \xi $ непрерывен в среднем квадратическом на $T$,
если $ \xi $ непрерывен в среднем квадратическом в каждой точке $t_0 \in T$.

\subsubsection*{Как определяются производная случайного процесса, интеграл случайного процесса?}

$ \xi $ дифференцируем в среднем квадратическом в точке $t_0$, если
$$ \exists L_2-\lim \limits_{t \to t_0}
    \frac{ \xi \left(t \right) - \xi \left( t_0 \right) }{t - t_0} =
  \xi' \left( t_0 \right).$$

Случайный процесс $ \xi $ интегрируем в среднем квадратическом на отрезке $ \left[ a, b \right] $,
если
$$ \exists L_2-\lim \limits_{ \left| \pi \right| \to 0}
    \sum \limits_{k = 0}^{n - 1} \xi \left( t_k \right) \Delta t_k,$$
где $ \pi $ --- разбиение: $a = t_0 < \dotsc < t_n = b$.

Модуль разбиения --- это максимумальная разность между соседними точками.

\subsubsection*{Как изменяются характеристики случайного процесса при дифференцировании,
                интегрировании?}

$ \xi \in L_2 \, : \, M \left| \xi \right|^2 < \infty, \,
  \left( \xi, \eta \right) = M \xi \overline{ \eta }, \,
  \left( \xi, \xi \right) = M \left| \xi \right|^2$.

\subsubsection*{Приведите условия непрерывности, дифференцируемости,
                интегрируемости случайного процесса в терминах его ковариационной функции.}

$ \xi $ непрерывен в среднем квадратическом на интервале $T$ тогда и только тогда,
когда $m \in C \left( T \right), \, K \in C \left( T \times T \right) $.

Случайный процесс $ \xi $ дифференцируем в среднем квадратическои в точке $t_0$
тогда и только тогда, когда $m$ дифференцируема в точке $t_0$ и
$$ \exists \lim \limits_{s, t \to t_0}
    \frac{K \left( t, s \right) - K \left( t, t_0 \right) - K \left( t_0, s \right) + K \left( t_0, t_0 \right) }{ \left( s - t_0 \right) \left( t - t_0 \right) }.$$

Непрерывный в среднем квадратическом на отрезке $ \left[ a, b \right] $ процесс $ \xi $
интегрируем в среднем квадратическом на этом отрезке.

\addcontentsline{toc}{section}{Аудиторные задачи}
\section*{Аудиторные задачи}

\subsubsection*{7.2}

\textit{Задание.}
Пусть $ \zeta_1, \dotsc, \zeta_n$ --- интегрируемые с квадратом случайные величины.
Докажите, что случайный процесс
$$ \xi \left( t \right) =
  \sum \limits_{k = 1}^n \zeta_k e^{kt}$$
имеет производную в среднем квадратическом и найдите её.

\textit{Решение.}
$ \zeta_1, \dotsc, \zeta_n \in L_2$, то есть есть $n$ величин, и процесс определяется как
$$ \xi \left( t \right) =
  \sum \limits_{k = 1}^n \zeta_l e^{kt}.$$
Нужно проверить, что этот процесс дифференцируем и найти его производную в $L_2$.

Время $t$ входит только в экспоненту, которую мы умеем дифференцировать.
Нужно проверить, что
$$ \left \Vert
    \frac{ \xi \left( t \right) - \xi \left( t_0 \right) }{t - t_0} -
    \sum \limits_{k = 1}^n \zeta_k ke^{kt_0} \right \Vert \to
  0,$$
когда $t \to t_0$.

Будем это проверять.
Можно $ \xi \left( t \right) $ расписать.

$ \xi \left( t \right) $ --- это сумма
$$ \left \Vert
    \frac{ \xi \left( t \right) - \xi \left( t_0 \right) }{t - t_0} -
    \sum \limits_{k = 1}^n \zeta_k k e^{kt_0} \right \Vert =
  \left \Vert
    \frac{ \sum \limits_{k = 1}^n \zeta_k e^{kt} - \sum \limits_{k = 1}^n \zeta_k e^{kt_0}}{t - t_0} -
    \sum \limits_{k = 1}^n \zeta_k ke^{kt_0} \right \Vert =$$
Во всех слагаемых есть сумма и $ \zeta_k$.
Так что приведём подобные, и будет одна сумма
$$= \left \Vert
    \sum \limits_{k = 1}^n \zeta_k \left( \frac{e^{kt} - e^{kt_0}}{t - t_0} - ke^{kt_0} \right)
  \right \Vert \leq$$
Первое слагаемое в скобках стремится к производной, а второе и есть производная.
Из разность стремится к нулю.
Используем неравенство треугольника
$$ \leq \sum \limits_{k = 1}^n
  \left \Vert \zeta_k \left( \frac{e^{kt} - e^{kt_0}}{t - t_0} - ke^{kt_0} \right) \right \Vert =$$
С помощью свойства
$ \left \Vert \alpha \xi \right \Vert \leq
  \left| \alpha \right| \cdot \left \Vert \xi \right \Vert $
коэффициент выносится из нормы
$$= \sum \limits_{k = 1}^n
    \left| \frac{e^{kt} - e^{kt_0}}{t - t_0} - ke^{kt_0} \right| \cdot
    \left \Vert \zeta_k \right \Vert \to$$
Под модулем стоят числа, которые сходятся к нулю, под нормой --- числа
$$ \to 0.$$
