\addcontentsline{toc}{chapter}{Занятие 7. $L_2$ теория}
\chapter*{Занятие 7. $L_2$ теория}

\addcontentsline{toc}{section}{Контрольные вопросы и задания}
\section*{Контрольные вопросы и задания}

\subsubsection*{Приведите опредедение непрерывного в среднем квадратическом случайного процесса.}

$ \xi $ непрерывен в среднем квадратическом в точке $t_0$, если
$$ \xi \left( t \right) \overset{L_2}{ \to }
  \xi \left( t_0 \right) $$
при $t \to t_0$,
то есть $M \left[ \xi \left( t \right) - \xi \left( t_0 \right) \right]^2 \to 0, \, t \to t_0$.

Случайный процесс $ \xi $ непрерывен в среднем квадратическом на $T$,
если $ \xi $ непрерывен в среднем квадратическом в каждой точке $t_0 \in T$.

\subsubsection*{Как определяются производная случайного процесса, интеграл случайного процесса?}

$ \xi $ дифференцируем в среднем квадратическом в точке $t_0$, если
$$ \exists L_2-\lim \limits_{t \to t_0}
    \frac{ \xi \left(t \right) - \xi \left( t_0 \right) }{t - t_0} =
  \xi' \left( t_0 \right).$$

Случайный процесс $ \xi $ интегрируем в среднем квадратическом на отрезке $ \left[ a, b \right] $,
если
$$ \exists L_2-\lim \limits_{ \left| \pi \right| \to 0}
    \sum \limits_{k = 0}^{n - 1} \xi \left( t_k \right) \Delta t_k,$$
где $ \pi $ --- разбиение: $a = t_0 < \dotsc < t_n = b$.

Модуль разбиения --- это максимумальная разность между соседними точками.

\subsubsection*{Как изменяются характеристики случайного процесса при дифференцировании,
                интегрировании?}

$ \xi \in L_2 \, : \, M \left| \xi \right|^2 < \infty, \,
  \left( \xi, \eta \right) = M \xi \overline{ \eta }, \,
  \left( \xi, \xi \right) = M \left| \xi \right|^2$.

\subsubsection*{Приведите условия непрерывности, дифференцируемости,
                интегрируемости случайного процесса в терминах его ковариационной функции.}

$ \xi $ непрерывен в среднем квадратическом на интервале $T$ тогда и только тогда,
когда $m \in C \left( T \right), \, K \in C \left( T \times T \right) $.

Случайный процесс $ \xi $ дифференцируем в среднем квадратическои в точке $t_0$
тогда и только тогда, когда $m$ дифференцируема в точке $t_0$ и
$$ \exists \lim \limits_{s, t \to t_0}
    \frac{K \left( t, s \right) - K \left( t, t_0 \right) - K \left( t_0, s \right) + K \left( t_0, t_0 \right) }{ \left( s - t_0 \right) \left( t - t_0 \right) }.$$

Непрерывный в среднем квадратическом на отрезке $ \left[ a, b \right] $ процесс $ \xi $
интегрируем в среднем квадратическом на этом отрезке.
