\addcontentsline{toc}{chapter}{Занятие 13. Разложение Вольда. Задача прогноза.}
\chapter*{Занятие 13. Разложение Вольда. Задача прогноза.}

\addcontentsline{toc}{section}{Контрольные вопросы и задания}
\section*{Контрольные вопросы и задания}

\subsubsection*{Приведите определение сингулярной и регулярной стационарной последовательности,
                приведите примеры.}

Стационарная последовательность называется сингулярной, если
\begin{equation*}
  H_{-\infty }^{ \xi } =
  \dotsc =
  H_0^{ \xi } =
  \dotsc,
\end{equation*}
то есть если они все просто совпадают между собой, и называется регулярной,
если $H_{-\infty }^{ \xi } = \left\{ 0 \right\} $.

Примеры:
\begin{enumerate}
  \item если $ \left\{ \xi_n \right\} $~---~это последовательность случайных колебаний, то
  \begin{equation*}
    \mu =
    \sum \limits_{k = 1}^m \delta_{ \lambda_k}.
  \end{equation*}
  Прогноз: $ \hat{ \xi }_n = \xi_n$.

  Сейчас $H_n^{ \xi } = \overline{LS \left\{ \xi_k \right\} }, \, n \in \mathbb{Z}$.

  Нет зависимости от $n$, следовательно, $ \dotsc = H_n^{ \xi } = H_{n + 1}^{ \xi } = \dotsc $.

  Так как $H_n^{ \xi }$ не меняется, то
  \begin{equation*}
    \bigcap \limits_{n \in \mathbb{Z}} H_n^{ \xi } =
    H_0^{ \xi };
  \end{equation*}
  \item последовательность белого шума $ \left\{ \xi_n, \, n \in \mathbb{Z} \right\} $.
  У неё есть спектральная плотность $p \equiv 1$ на $ \left[ -\pi, \pi \right] $.
  В этом случае прогноза вообще никакого нет.
  Вместо проекции получаем $ \hat{ \xi }_n = 0$.

  Сейчас $H_n^{ \xi } = \overline{LS \left\{ \xi_k, \, k \leq n \right\} }$.

  Так как последовательность белого шума~---~ортонональные случайные величины,
  то $ \xi_{n + 1} \perp H_n^{ \xi }$.

  Следовательно, $H_n^{ \xi } \subset H_{n + 1}^{ \xi }$ (включение строгое).

  $H_n^{ \xi }$~---~это прошлое до момента времени $n$.

  Тогда
  \begin{equation*}
    \bigcap \limits_{n \in \mathbb{Z}} H_n^{ \xi } =
    \left\{ 0 \right\}.
  \end{equation*}

  Докажем это.
  Возьмём
  \begin{equation*}
    \zeta =
    \bigcap \limits_{n \in \mathbb{Z}} H_n^{ \xi } \equiv
    H_{-\infty }^{ \xi }.
  \end{equation*}
  Если $ \zeta $ принадлежит пересечению, то $ \zeta \in H_0^{ \xi }$.
  Раз так, то $ \zeta $ должно представляться как предел линейной комбинации
  \begin{equation*}
    \zeta =
    \lim \limits_{m \to \infty } \sum \limits_{k = -m}^0 c_{km} \zeta_k.
  \end{equation*}

  Замыканием такой линейной комбинации и есть $H_0^{ \xi }$.

  Одновременно с этим $ \zeta \in H_{-1}^{ \xi }$.
  Следовательно, $ \zeta \perp \xi_0$.
  Можем сделать заключение, что $ \zeta \in H_{-2}^{ \xi } \Rightarrow \zeta \perp \xi_{-1}$.

  Таким образом, $ \forall k \leq 0 \; : \; \zeta \perp \xi_k$.
  Отсюда
  \begin{equation*}
    M \left| \zeta \right|^2 =
    \lim \limits_{m \to \infty } M \zeta \cdot \overline{ \sum \limits_{k = -m}^0 c_{km} \zeta_k} =
    0
  \end{equation*}
  (так как $ \zeta $ ортогональная ко всем $ \xi_k$).

  Таким образом, $H_{-\infty }^{ \xi } = \left\{ 0 \right\} $.
\end{enumerate}

\subsubsection*{Опишите пространства, связанные со стационарной последовательностью.}

Будем использовать
$ \forall n \in \mathbb{Z} \; : \;
  H_n^{ \xi } = \overline{LS \left\{ \xi_k, \, k \leq n \right\} }$~---~замыкание
в среднем квадратическом линейной оболочки $ \left\{ \xi_k, \, k \leq n \right\} $.

Получили $H_n^{ \xi }$~---~подпространство пространства
$L_2 \left( \Omega, \mathcal{F}, P \right) $.

Пусть $Q_n$~---~это проектор на $H_n$.

\subsubsection*{Что такое разложение Вольда?}

Для произвольной стационарной последовательности $ \left\{ \xi_n \right\} $
существует регулярная последовательность $ \left\{ \xi_n' \right\} $
и сингулярная последовательность $ \left\{ \xi_n'' \right\} $ такие, что
\begin{enumerate}
  \item $ \xi_n = \xi_n' + \xi_n''$;
  \item $ \xi_n' \perp \xi_m'', \qquad \forall n, m \in \mathbb{Z}$;
  \item $ \xi_n', \xi_n'' \in H_n^{ \xi }$.
\end{enumerate}

Это представление называется разложением Вольда.

\subsubsection*{Как определяются прогноз и погрешность прогноза $ \xi_n, \, n \geq 1$ при
                $ \xi^0 = \left( \dotsc, \xi_{-1}, \xi_0 \right) $ с помощью разложения Вольда?}

Решение задачи прогноза:
\begin{equation*}
  Q_{n_0} \xi_n =
  \sum \limits_{k = n - n_0}^{ \infty } \alpha_k e_{n - k}.
\end{equation*}

Можем посчитать ошибку прогноза:
\begin{equation*}
  \left \Vert \xi_n - Q_{n_0} \xi_n \right \Vert^2 =
  \left \Vert \sum \limits_{k = 0}^{n - n_0 - 1} \alpha_k e_{n - k} \right \Vert^2 =
  \sum \limits_{k = 0}^{n - n_0 - 1} \left| \alpha_k \right|^2.
\end{equation*}

\subsubsection*{Запишите формулу для погрешности прогноза в терминах спектральной плотности.}

\begin{equation*}
  \sigma^2 =
  e^{ \frac{1}{2 \pi } \int \limits_{-\pi }^{ \pi } ln \, p \left( \lambda \right) d \lambda }.
\end{equation*}

\addcontentsline{toc}{section}{Аудиторные задачи}
\section*{Аудиторные задачи}

\addcontentsline{toc}{section}{Домашнее задание}
\section*{Домашнее задание}
