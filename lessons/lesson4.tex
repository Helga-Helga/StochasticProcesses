\addcontentsline{toc}{chapter}{Занятие 4. Гауссовские процессы}
\chapter*{Занятие 4. Гауссовские процессы}

\addcontentsline{toc}{section}{Контрольные вопросы и задания}
\section*{Контрольные вопросы и задания}

\subsubsection*{Приведите определение гауссовского процесса.}

Процесс $ \left\{ X \left( t \right), \, t \in T \right\} $ --- гауссовский, если
$$ \sum \limits_{k = 1}^n \lambda_k X \left( t_k \right) $$
--- гауссовская случайная величина $ \forall \vec{ \lambda } \in \mathbb{R}^n$ и
$ \forall t_1, \dotsc, t_n \in T$.

Эквивалентно: $ \left( X \left( t_1 \right), \dotsc, X \left( t_n \right) \right)^T$ ---
гауссовский вектор.

\subsubsection*{Запишите плотность конечномерных распределений гауссовского процесса.}

$$p =
  \frac{1}{ \left( \sqrt{2 \pi} \right)^n} \cdot \frac{1}{ \sqrt{det \, A}} \cdot
  e^{-\frac{1}{2} \left[ A^{-1} \left( \vec{x} - \vec{a} \right), \vec{x} - \vec{a} \right] },$$
если $det \, A > 0$.
Квадратные скобки в степени экспоненты --- это скалярное произведение,
или квадратичная форма матрицы, обратной к ковариации.

\subsubsection*{Приведите определение и сформулируйте основные свойства ковариационной функции.}

$K \left( t, s \right) =
  cov \left[ X \left( t \right), X \left( s \right) \right] $.

Гауссовский процесс существует, из теомеры Колмогорова, с функциями $m$ и $K$ тогда и только тогда,
когда функция $K \left( s, t \right) = K \left( t, s \right) $ --- симметричная, и $K$ ---
неотрицательно определённая, то есть
$$ \sum \limits_{k, j = 1}^n c_k c_j K \left( t_k, t_j \right) \geq
  0.$$

Тут неравенство возможно для любых $c_1, \dotsc, c_n, t_1, \dotsc, t_n$.
