\addcontentsline{toc}{chapter}{Занятие 15. Цепи Маркова. Период состояний.
                                Инвариантное распределение.}
\chapter*{Занятие 15. Цепи Маркова. Период состояний. Инвариантное распределение.}

\addcontentsline{toc}{section}{Контрольные вопросы и задания}
\section*{Контрольные вопросы и задания}

\subsubsection*{Приведите определение цепи Маркова.}

Последовательность дискретных случайных величин $ \left\{ x_n \right\}_{n \geq 0}$
называется простой цепью Маркова (с дискретным временем), если
\begin{equation*}
  P \left( x_{n + 1} = i_{n + 1} \; \middle| \; x_n = i_n, \dotsc, x_0 = i_0 \right) =
  P \left( x_{n + 1} = i_{n + 1} \; \middle| \; x_n = i_n \right).
\end{equation*}

\subsubsection*{Что называется переходными вероятностями цепи Маркова?}

Матрица $P \left( n \right) $,
где $P_{ij} \left( n \right) = P \left( x_{n + 1} = i_{n + 1} \; \middle| \; x_n = i \right) $,
называется матрицей переходных вероятностей на $n$-м шаге.

\subsubsection*{Как вычисляются переходные вероятности цепи Маркова за $n$ шагов?}

Матрица переходных вероятностей за $n$ шагов однородной цепи Маркова есть $n$-я
степень матрицы переходных вероятностей за 1 шаг.

\subsubsection*{Запишите уравнение Колмогорова-Чепмена.}

$P \left( x_n - i_n \; \middle| \; x_0 = i_0 \right) =
  \left( P^n \right)_{i_0, i_n}$.

\subsubsection*{Опишите, как классифицируются состояния цепи Маркова.}

Группы состояний марковской цепи (подмножества вершин графа переходов),
которым соответствуют тупиковые вершины диаграммы порядка графа переходов,
называются эргодическими классами цепи.
Состояния, которые находятся в эргодических классах, называются существенными,
а остальные~---~несущественными.
Поглощающее состояние является частным случаем эргодического класса.
Тогда попав в такое состояние, процесс прекратится.

\subsubsection*{Что называется периодом состояния?}

Пусть дана однородная цепь Маркова с дискретным временем $ \left\{ x_n \right\}_{n \geq 0}$
с матрицей переходных вероятностей $P$.
В частности, для любого $n \in \mathbb{N}$,
матрица $P^n = \left( p_{ij}^{ \left( n \right) } \right) $
является матрицей переходных вероятностей за $n$ шагов.
Рассмотрим последовательность $p_{jj}^{ \left( n \right) }, \, n \in \mathbb{N}$.
Число
\begin{equation*}
  d \left( j \right) =
  gcd \left( n \in \mathbb{N} \; \middle| \; p_{jj}^{ \left( n \right) } > 0 \right),
\end{equation*}
где $gcd$ обозначает наибольший общий делитель, называется периодом состояния $j$.

\subsubsection*{Сформулируйте утверждение про периоды сообщающихся состояний.}

Периоды сообщающихся состояний совпадают:
$ \left( i \leftrightarrow j \right) \Rightarrow \left( d \left( i \right) =
  d \left( j \right) \right) $.

\subsubsection*{Сформулируйте теорему про ассимптотическое поведение переходных вероятностей
                $p_{ij}^{ \left( n \right) }$ при $n \to \infty $.}

При $n \to \infty $ матрица $P^n$ стремится к строго положительной матрице,
у которой все строки совпадают с равновесным распределением.

\subsubsection*{Дайте определение инвариантного распределения цепи Маркова.}

Будем предполагать,
что $ \sigma $-алгебра $ \varepsilon $ порождена счётным семейством подмножеств $E$.

Всякая $ \sigma $-конечная мера $ \pi \left( \cdot \right) $, удовлетворяющая уравнению
\begin{equation*}
  \pi \left( A \right) = \int \limits_{E} \pi \left( dx \right) P \left( x, A \right), \,
  A \in \varepsilon
\end{equation*}
и условию $ \pi \left( A \right) < \infty $ хотя бы для одного множества $A \in \varepsilon^+$,
называется инвариантной.

\addcontentsline{toc}{section}{Аудиторные задачи}
\section*{Аудиторные задачи}
