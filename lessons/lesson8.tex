\addcontentsline{toc}{chapter}{Занятие 8. Стационарные случайные процессы}
\chapter*{Занятие 8. Стационарные случайные процессы}

\addcontentsline{toc}{section}{Контрольные вопросы и задания}
\section*{Контрольные вопросы и задания}

\subsubsection*{Приведите определение случайного процесса, стационарного в широком смысле.}

$ \xi \left( t \right), \, t \in T$ называется стационарным в широком смысле (стационарным), если
\begin{enumerate}
  \item $m \left( t \right) \equiv m \, \left( const \right) $;
  \item $K \left( t, s \right) = K \left( t + r, s + r \right), \, \forall t, s, r \in T$.
  Это означает, что ковариационная функция --- это сейчас функция разности аргументов,
  то есть $K \left( t, s \right) = k \left( t - s \right) $.
\end{enumerate}

\subsubsection*{Приведите определение ковариационной функции случайного процесса.}

$K \left( t, s \right) =
  M \left[ \xi \left( t \right) - m \left( t \right) \right] \cdot
  \left[ \xi \left( s \right) - m \left( s \right) \right], \,
  t, s \in T$.

\subsubsection*{Сформулируйте теорему Бохнера про спектральное изображения ковариационной функции
                стационарного в широком смысле случайного процесса.}

Пусть $ \xi \left( t \right), \, t \in \mathbb{R}$ ---
это стационарный и непрерывный в среднем квадратическом случайный процесс.
Тогда существует конечная мера $ \mu $ на $ \mathbb{R}$ такая, что
$$k \left( t \right) =
  \int \limits_{-\infty}^{+\infty} e^{it \lambda } \mu \left( d \lambda \right),$$
мера $ \mu $ определяется единственным образом.

\subsubsection*{Что называется спектральной функцией, спектральной плотностью стационарного в
                широком смысле случайного процесса?}

$ \mu $ называеься спектральной мерой, а если есть плотность $p$, то $p$ ---
это спектральная плотность.

\addcontentsline{toc}{section}{Аудиторные задачи}
\section*{Аудиторные задачи}

\subsubsection*{8.2}

\textit{Задание.}
Пусть $A, \eta, \varphi $ --- независимые случайные величины,
причём $ \varphi $ имеет равномерное распределение на отрезке $ \left[ 0, 2 \pi \right] $.
Докажите, что процесс
$ \left\{
    \xi \left( t \right) = A \cos \left( t \eta + \varphi \right), \, t \in \mathbb{R}
  \right\} $
является стационарным в широком смысле.

\textit{Решение.}
$$M \xi \left( t \right) =
  M \left[ A \cos \left( \eta t + \varphi \right) \right] =
  MA \cdot M \cos \left( \eta t + \varphi \right) =$$
Распишем косинус суммы
$$= MA \cdot M \left[
    \cos \left( \eta t \right) \cos \varphi - \sin \left( \eta t \right) \sin \varphi \right] =$$
Все слагаемые и множители независимы
$$= MA \cdot M \cos \left( \eta t \right) \cdot M \cos \varphi -
  MA \cdot M \sin \left( \eta t \right) \cdot M \sin \varphi =$$
Сгруппируем множители
$$= M \left[ A \cos \left( \eta t \right) \right] \cdot M \cos \varphi -
  M \left[ A \sin \left( \eta t \right) \right] \cdot M \sin \varphi =$$
Запишем математическое ожидание $ \varphi $ через интеграл от плотности равномерного распределения
$$= M \left[ A \cos \left( \eta t \right) \right]
  \int \limits_{ \mathbb{R}} \cos x \cdot p \left( x \right) dx -
  M \left[ A \sin \left( \eta t \right) \right]
  \int \limits_{ \mathbb{R}} \sin x \cdot p \left( x \right) dx =$$
Подставим плотность
$$= M \left[ A \cos \left( \eta t \right) \right] \int \limits_0^{2 \pi } \cos xdx -
  M \left[ A \sin \left( \eta t \right) \right] \int \limits_0^{2 \pi } \sin xdx =$$
Возьмём интегралы
$$= M \left[ A \cos \left( \eta t \right) \right] \cdot \left. \sin x \right|_0^{2 \pi } -
  M \left[ A \sin \left( \eta t \right) \right] \cdot
  \left. \left( - \cos x \right) \right|_0^{2 \pi } =
  0.$$

Ковариационная функция
$$K \left( t, s \right) =
  M \xi \left( t \right) \xi \left( s \right) =
  M A^2 \cdot M \cos \left( \eta t + \varphi \right) \cdot \cos \left( \eta s + \varphi \right) =$$
Распишем произведение косинусов
$$= \frac{MA^2}{2} \cdot \left\{
  M \cos \left[ \eta \left( t + s \right) + 2 \varphi \right] +
  M \cos \left[ \eta \left( t - s \right) \right] \right\} =$$
Первое слагаемое равно нулю
$$= \frac{MA^2}{2} \cdot M \cos \left[ \eta \left( t - s \right) \right].$$
Значит, процесс стационарный.

\addcontentsline{toc}{section}{Домашнее задание}
\section*{Домашнее задание}

\subsubsection*{8.10}

\textit{Задание.}
Докажите, что сумма независимых стационарных в широком смысле процессов является стационарным в
широком смысле процессом.

\textit{Решение.}
Пусть $ \left\{ \xi_i \left( t \right), \, t \in T \right\}, \, i = \overline{1, n}$ ---
независимые стационарные в широком смысле процессы.
Нужно доказать, что процесс
$$ \left\{
  \eta \left( t \right) = \sum \limits_{i = 1}^n \xi_i \left( t \right), \, t \in T \right\} $$
является стационарным в широком смысле.

$$M \eta \left( t \right) =
  M \sum \limits_{i = 1}^n \xi_i \left( t \right) =
  \sum \limits_{i = 1}^n M \xi_i \left( t \right) =
  \sum \limits_{i = 1}^n m_i =
  m =
  const.$$

Ковариационная функция
$$K \left( t, s \right) =
  cov \left[ \eta \left( t \right), \eta \left( s \right) \right] =
  cov \left[
    \sum \limits_{i = 1}^n \xi_i \left( t \right), \sum \limits_{j = 1}^n \xi_j \left( s \right)
  \right] =$$
Вынесем суммы, так как ковариация --- линейная функция
$$= \sum \limits_{i = 1}^n
    \sum \limits_{j = 1}^n cov \left[ \xi_i \left( t \right), \xi_j \left( s \right) \right] =
  \sum \limits_{i = 1}^n cov \left[ \xi_i \left( t \right), \xi_i \left( s \right) \right] =
  \sum \limits_{i = 1}^n K_i \left( t, s \right) =$$
Все $ \xi_i \left( t \right) $ стационарные,
поэтому их ковариационные функции зависят только от разности аргументов
$$= \sum \limits_{i = 1}^n k_i \left( t - s \right) =
  k \left( t - s \right).$$
Значит, процесс стационарный.

\subsubsection*{8.11}

\textit{Задание.}
Пусть $ \xi_1, \xi_2$ --- независимые одинаково распределённые случайные величины,
которые принимают значения $+1$ и $-1$ с вероятностью $1 / 2$.
Докажите, что процесс
$ \left\{
  \xi \left( t \right) = \xi_1 \cos \lambda t + \xi_2 \sin \lambda t, \, t \in \mathbb{R} \right\} $
является стационарным в широком смысле.

\textit{Решение.}
$$M \xi \left( t \right) =
  M \left( \xi_1 \cos \lambda t + \xi_2 \sin \lambda t \right) =
  M \left( \xi_1 \cos \lambda t \right) + M \left( \xi_2 \sin \lambda t \right) =$$
Вынесем константы
$$= \cos \lambda t \cdot M \xi_1 + \sin \lambda t \cdot M \xi_2 =
  \cos \lambda t \cdot \left( 1 \cdot \frac{1}{2} - 1 \cdot \frac{1}{2} \right) +
  \sin \lambda t \cdot \left( 1 \cdot \frac{1}{2} - 1 \cdot \frac{1}{2} \right) =
  0.$$

Ковариационная функция
$$K \left( t, s \right) =
  M \xi \left( t \right) \xi \left( s \right) =
  M \left[
    \left( \xi_1 \cos \lambda t + \xi_2 \sin \lambda t \right)
    \left( \xi_1 \cos \lambda s + \xi_2 \sin \lambda s \right) \right] =$$
Перемножим скобки
\begin{gather*}
  = \cos \lambda t \cdot \cos \lambda s \cdot M \xi_1^2 +
  \cos \lambda t \cdot \sin \lambda s \cdot M \left( \xi_1 \xi_2 \right) +
  \sin \lambda t \cdot \cos \lambda s \cdot M \left( \xi_2 \xi_1 \right) + \\
  + \sin \lambda t \cdot \sin \lambda s \cdot M \xi_2^2 =
  \cos \lambda t \cdot \cos \lambda s + \sin \lambda t \cdot \sin \lambda s =
  \cos \left[ \lambda \left( t - s \right) \right] = \\
  = k \left( t - s \right).
\end{gather*}
Значит, процесс стационарный.
