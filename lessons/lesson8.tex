\addcontentsline{toc}{chapter}{Занятие 8. Стационарные случайные процессы}
\chapter*{Занятие 8. Стационарные случайные процессы}

\addcontentsline{toc}{section}{Контрольные вопросы и задания}
\section*{Контрольные вопросы и задания}

\subsubsection*{Приведите определение случайного процесса, стационарного в широком смысле.}

$ \xi \left( t \right), \, t \in T$ называется стационарным в широком смысле (стационарным), если
\begin{enumerate}
  \item $m \left( t \right) \equiv m \, \left( const \right) $;
  \item $K \left( t, s \right) = K \left( t + r, s + r \right), \, \forall t, s, r \in T$.
  Это означает, что ковариационная функция --- это сейчас функция разности аргументов,
  то есть $K \left( t, s \right) = k \left( t - s \right) $.
\end{enumerate}

\subsubsection*{Приведите определение ковариационной функции случайного процесса.}

$K \left( t, s \right) =
  M \left[ \xi \left( t \right) - m \left( t \right) \right] \cdot
  \left[ \xi \left( s \right) - m \left( s \right) \right], \,
  t, s \in T$.

\subsubsection*{Сформулируйте теорему Бохнера про спектральное изображения ковариационной функции
                стационарного в широком смысле случайного процесса.}

Пусть $ \xi \left( t \right), \, t \in \mathbb{R}$ ---
это стационарный и непрерывный в среднем квадратическом случайный процесс.
Тогда существует конечная мера $ \mu $ на $ \mathbb{R}$ такая, что
$$k \left( t \right) =
  \int \limits_{-\infty}^{+\infty} e^{it \lambda } \mu \left( d \lambda \right),$$
мера $ \mu $ определяется единственным образом.

\subsubsection*{Что называется спектральной функцией, спектральной плотностью стационарного в
                широком смысле случайного процесса?}

$ \mu $ называеься спектральной мерой, а если есть плотность $p$, то $p$ ---
это спектральная плотность.
