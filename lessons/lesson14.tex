\addcontentsline{toc}{chapter}{Занятие 14. Цепи Маркова. Классификация состояний.}
\chapter*{Занятие 14. Цепи Маркова. Классификация состояний.}

\addcontentsline{toc}{section}{Контрольные вопросы и задания}
\section*{Контрольные вопросы и задания}

\subsubsection*{Приведите определение цепи Маркова.}

Последовательность дискретных случайных величин $ \left\{ x_n \right\}_{n \geq 0}$
называется простой цепью Маркова (с дискретным временем), если
\begin{equation*}
  P \left( x_{n + 1} = i_{n + 1} \; \middle| \; x_n = i_n, \dotsc, x_0 = i_0 \right) =
  P \left( x_{n + 1} = i_{n + 1} \; \middle| \; x_n = i_n \right).
\end{equation*}

\subsubsection*{Что называется переходными вероятностями цепи Маркова?}

Матрица $P \left( n \right) $,
где $P_{ij} \left( n \right) = P \left( x_{n + 1} = i_{n + 1} \; \middle| \; x_n = i \right) $,
называется матрицей переходных вероятностей на $n$-м шаге.

\subsubsection*{Как вычисляются переходные вероятности цепи Маркова за $n$ шагов?}

Матрица переходных вероятностей за $n$ шагов однородной цепи Маркова есть $n$-я
степень матрицы переходных вероятностей за 1 шаг.

\subsubsection*{Запишите уравнение Колмогорова-Чепмена.}

$P \left( x_n - i_n \; \middle| \; x_0 = i_0 \right) =
  \left( P^n \right)_{i_0, i_n}$.

\subsubsection*{Опишите, как классифицируются состояния цепи Маркова.}

Группы состояний марковской цепи (подмножества вершин графа переходов),
которым соответствуют тупиковые вершины диаграммы порядка графа переходов,
называются эргодическими классами цепи.
Состояния, которые находятся в эргодических классах, называются существенными,
а остальные~---~несущественными.
Поглощающее состояние является частным случаем эргодического класса.
Тогда попав в такое состояние, процесс прекратится.
