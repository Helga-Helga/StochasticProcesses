\addcontentsline{toc}{chapter}{Занятие 14. Цепи Маркова. Классификация состояний.}
\chapter*{Занятие 14. Цепи Маркова. Классификация состояний.}

\addcontentsline{toc}{section}{Контрольные вопросы и задания}
\section*{Контрольные вопросы и задания}

\subsubsection*{Приведите определение цепи Маркова.}

Последовательность дискретных случайных величин $ \left\{ x_n \right\}_{n \geq 0}$
называется простой цепью Маркова (с дискретным временем), если
\begin{equation*}
  P \left( x_{n + 1} = i_{n + 1} \; \middle| \; x_n = i_n, \dotsc, x_0 = i_0 \right) =
  P \left( x_{n + 1} = i_{n + 1} \; \middle| \; x_n = i_n \right).
\end{equation*}

\subsubsection*{Что называется переходными вероятностями цепи Маркова?}

Матрица $P \left( n \right) $,
где $P_{ij} \left( n \right) = P \left( x_{n + 1} = i_{n + 1} \; \middle| \; x_n = i \right) $,
называется матрицей переходных вероятностей на $n$-м шаге.

\subsubsection*{Как вычисляются переходные вероятности цепи Маркова за $n$ шагов?}

Матрица переходных вероятностей за $n$ шагов однородной цепи Маркова есть $n$-я
степень матрицы переходных вероятностей за 1 шаг.

\subsubsection*{Запишите уравнение Колмогорова-Чепмена.}

$P \left( x_n - i_n \; \middle| \; x_0 = i_0 \right) =
  \left( P^n \right)_{i_0, i_n}$.

\subsubsection*{Опишите, как классифицируются состояния цепи Маркова.}

Группы состояний марковской цепи (подмножества вершин графа переходов),
которым соответствуют тупиковые вершины диаграммы порядка графа переходов,
называются эргодическими классами цепи.
Состояния, которые находятся в эргодических классах, называются существенными,
а остальные~---~несущественными.
Поглощающее состояние является частным случаем эргодического класса.
Тогда попав в такое состояние, процесс прекратится.

\addcontentsline{toc}{section}{Аудиторные задачи}
\section*{Аудиторные задачи}

\subsubsection*{14.2}

\textit{Задание.}
Подбрасывается игральный кубик.
Выясните
образует ли последовательность $ \left\{ \xi_n \right\}_{n \geq 1}$ однородную цепь Маркова, если
\begin{enumerate}[label=\alph*)]
  \item $ \xi_n$~---~это наибольшее из чисел, которые выпали в первых $n$ подбрасываниях;
  \item $ \xi_n$~---~это количество шестёрок, которые выпали в первых $n$ подбрасываниях.
\end{enumerate}

\textit{Решение.}
\begin{enumerate}[label=\alph*)]
  \item $ \xi_n = \max \left( x_1, \dotsc, x_n \right) $,
  где $x_1, x_2, \dotsc, x_n$~---~это результаты подбрасываний кубика
  (принимают значения $1, \dotsc, 6$).
  Нужно проверить, что вероятность зависит только от $j$ и $i_n$.
  Попробуем $ \xi_{n + 1}$ переписать через $ \xi_n$.
  Напишем рекуррентное соотношение $ \xi_{n + 1} = \max \left( \xi_n, x_{n + 1} \right) $.
  Подставим это в формулу
  \begin{gather*}
    P \left( \xi_{n + 1} = j \; \middle| \; \xi_1 = i_1, \dotsc, \xi_n = i_n \right) = \\
    = P \left\{
      \max \left( \xi_n, x_{n + 1} \right) = j \; \middle| \; \xi_1 = i_1, \dotsc, \xi_n = i_n
    \right\} =
  \end{gather*}
  Знаем, что $ \xi_n = i_n$.
  Тогда
  \begin{equation*}
    = P \left\{
      \max \left( i_n, x_{n + 1} \right) = j \; \middle| \; \xi_1 = i_1, \dotsc, \xi_n = i_n
    \right\} =
  \end{equation*}
  Случайная величина $ \max \left( i_n, x_{n + 1} \right) $ зависит только от $x_{n + 1}$.
  Условие зависит только от $x_1, \dotsc, x_n$.
  Событие и условие независимы.
  Эта вероятность станосится безусловной
  \begin{equation*}
    = P \left\{ \max \left( i_n, x_{n + 1} \right) = j \right\} =
  \end{equation*}
  Вывод: вероятность зависит только от $i_n$ и $j$, так что это марковская цепь.
  Посчитаем вероятность
  \begin{equation*}
    = \begin{cases}
      0, \qquad i_n > j, \\
      P \left( x_1 \leq j \right), \qquad i_n = j, \\
      P \left( x_1 = j \right), \qquad i_n < j
  \end{cases} =
  \end{equation*}
  Случайная величина $x_1$ принимает значения с вероятностями $6^{-1}$.
  Таким образом, получаем
  \begin{equation*}
    = \begin{cases}
        0, \qquad i_n > j, \\
        \frac{j}{6}, \qquad i_n = j, \\
        \frac{1}{6}, \qquad i_n < j.
      \end{cases}
  \end{equation*}

  Значит, можно теперь нарисовать матрицу переходных вероятностей
  \begin{equation*}
    \bordermatrix{i_n \ j & 1 & 2 & 3 & 4 & 5 & 6 \cr
                  1 & \frac{1}{6} & \frac{1}{6} & \frac{1}{6} & \frac{1}{6} & \frac{1}{6} & \frac{1}{6}\cr
                  2 & 0 & \frac{2}{6} & \frac{1}{6} & \frac{1}{6} & \frac{1}{6} & \frac{1}{6} \cr
                  3 & 0 & 0 & \frac{3}{6} & \frac{1}{6} & \frac{1}{6} & \frac{1}{6} \cr
                  4 & 0 & 0 & 0 & \frac{4}{6} & \frac{1}{6} & \frac{1}{6} \cr
                  5 & 0 & 0 & 0 & 0 & \frac{5}{6} & \frac{1}{6} \cr
                  6 & 0 & 0 & 0 & 0 & 0 & 1 \cr}
  \end{equation*}

  Получилая верхнедиагональная матрица.
  В каждой строчке сумма~---~единица.
  \item Нужно начинать с того, что написать формулу для $ \xi_n$, чтобы представить его через $x$.

  \begin{equation*}
    \xi_n =
    \sum \limits_{i = 1}^n \mathbbm{1} \left\{ x_i = 6 \right\} =
  \end{equation*}
  Видели, что удобно иметь рекурентное соотношение
  \begin{equation*}
    = \sum \limits_{i = 1}^{n - 1} \mathbbm{1} \left\{ x_i = 6 \right\} +
    \mathbbm{1} \left\{ x_n = 6 \right\} =
    \xi_{n - 1} + \mathbbm{1} \left\{ \xi_n = 6 \right\}.
  \end{equation*}
  Проверим, что это марковская цепь.

  Подставим в вероятность выражение для $ \xi_n$ через $ \xi_{n - 1}$, то есть
  \begin{gather*}
    P \left( \xi_n = j \; \middle| \; \xi_{n - 1} = i_{n - 1}, \dotsc, \xi_1 = i_1 \right) = \\
    = P \left(
      \xi_{n - 1} + \mathbbm{1} \left\{ x_n = 6 \right\} = j \; \middle| \;
      \xi_{n - 1} = i_{n - 1}, \dotsc, \xi_1 = i_1
    \right) = \\
    = P \left(
      i_{n - 1} + \mathbbm{1} \left\{ x_n = 6 \right\} = j  \; \middle| \;
      \xi_{n - 1} = i_{n - 1}, \dotsc, \xi_1 = i_1
    \right) =
  \end{gather*}
  Событие и условие независимы, опять вероятность безусловная
  \begin{equation*}
    = P \left( i_{n - 1} + \mathbbm{1} \left\{ x_n = 6 \right\} = j \right) =
  \end{equation*}
  Из этого уже следует, что это марковская цепь
  \begin{equation*}
    = P \left( \mathbbm{1} \left\{ x_1 = 6 \right\} = j - i_{n - 1} \right) =
    \begin{cases}
      \frac{1}{6}, \qquad j = i_{n - 1} + 1, \\
      \frac{5}{6}, \qquad j = i_{n - 1}, \\
      0, \qquad in \, all \, other \, cases.
    \end{cases}
  \end{equation*}

  У нас получилась марковская цепь с такими переходными вероятностями
  \begin{equation*}
    \bordermatrix{~ & 0 & 1 & 2 & 3 & \dotsc \cr
                  0 & \frac{5}{6} & \frac{1}{6} & 0 & 0 & \dotsc \cr
                  1 & 0 & \frac{5}{6} & \frac{1}{6} & 0 & \dotsc \cr
                  2 & 0 & 0 & \frac{5}{6} & \frac{1}{6} & \dotsc \cr
                  3 & 0 & 0 & 0 & \frac{5}{6} & \dotsc \cr
                  \dotsc & \dotsc & \dotsc & \dotsc & \dotsc & \dotsc \cr}
  \end{equation*}

  Сейчас матрица бесконечна.
\end{enumerate}
