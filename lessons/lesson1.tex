\addcontentsline{toc}{chapter}{Занятие 1. Основные понятия теории вероятностей}
\chapter*{Занятие 1. Основные понятия теории вероятностей}

\addcontentsline{toc}{section}{Аудиторные задачи}
\section*{Аудиторные задачи}

\subsubsection*{1.1}

\textit{Задание.}
Случайная величина $ \xi $ имеет плотность распределения
\begin{equation*}
  p_{ \xi } \left( x \right) =
  \begin{cases}
    3e^{-Cx}, \qquad x > 0, \\
    0, \qquad x \leq 0,
  \end{cases}
\end{equation*}
где $C > 0$~---~неизвестный параметр.
Найдите:
\begin{enumerate}[label=\alph*)]
  \item неизвестный параметр $C$;
  \item условную вероятность
  $P \left( \xi > t \; \middle| \; \xi > s \right), \, t > s$,
  \item функцию распределения случайной величины $ \xi $;
  \item плотность распределения случайной величины $ \eta = e^{ \xi }$.
\end{enumerate}

\textit{Решение.}
\begin{enumerate}[label=\alph*)]
  \item Находим неизвестный параметр $C$, который входит в плотность,
  из условия нормировки.

  Плотность определена, когда $x > 0$.

  Условие нормировки выполняется для всякой плотности
  \begin{equation*}
    1 =
    \int \limits_{-\infty }^{+\infty } p_{ \xi } \left( x \right) dx =
    \int \limits_0^{+\infty } 3e^{-Cx} dx =
    \left. -\frac{3}{C} \cdot e^{-Cx} \right|_0^{+\infty } =
    \frac{3}{C}.
  \end{equation*}

  Отсюда следует, что $C = 3$;
  \item по определению условной вероятности
  \begin{equation*}
    P \left( \xi > t \; \middle| \; \xi > s \right) =
    \frac{P \left( \xi > t, \xi > s \right) }{P \left( \xi > s \right) } =
  \end{equation*}
  По условию $t > s$, поэтому в числителе одно из уловий лишнее
  \begin{equation*}
    = \frac{P \left( \xi > t \right) }{P \left( \xi > s \right) } =
    \frac{ \int \limits_t^{+\infty } 3xe^{-3x} \cdot \mathbbm{1} \left\{ x > 0 \right) dx}{ \int \limits_s^{+\infty } 3xe^{-3x} \cdot \mathbbm{1} \left\{ x > 0 \right\} } =
  \end{equation*}
  Пусть $t, s > 0$.
  Тогда
  \begin{equation*}
    = \frac{ \int \limits_t^{+\infty } xe^{-3x} dx}{ \int \limits_s^{+\infty } xe^{-3x} dx}.
  \end{equation*}
  Возьмём интеграл по частям.
  Пусть
  \begin{equation*}
    u = x, \,
    dv = e^{-3x} dx, \,
    v = \int e^{-3x} dx = - \frac{1}{3} e^{-3x}, \,
    du = dx.
  \end{equation*}
  Тогда
  \begin{equation*}
    \int xe^{-3x} dx =
    -\frac{1}{3} \cdot xe^{-3x} + \frac{1}{3} \int e^{-3x} dx =
    -\frac{1}{3} \cdot xe^{-3x} - \frac{1}{9} \cdot e^{-3x}.
  \end{equation*}
  Подставим полученное выражение в дробь
  \begin{equation*}
    = \left( \left. -\frac{1}{3} \cdot xe^{-3x} \right|_t^{+\infty } - \left. \frac{1}{9} \cdot e^{-3x} \right|_t^{+\infty } \right) \cdot
    \frac{1}{ \left. -\frac{1}{3} \cdot xe^{-3x} \right|_s^{+\infty } - \left. \frac{1}{9} \cdot e^{-3x} \right|_s^{+\infty }} =
  \end{equation*}
  Подставим пределы интегрирования
  \begin{equation*}
    = \frac{ \frac{1}{3} \cdot te^{-3t} + \frac{1}{9} \cdot e^{-3t}}{ \frac{1}{3} \cdot se^{-3s} + \frac{1}{9} \cdot e^{-3s}} =
    \frac{e^{-3t} \left(3t + 1 \right) }{e^{-3s} \left( 3s + 1 \right) } =
    e^{3 \left( s - t \right) } \cdot \frac{3t + 1}{3s + 1};
  \end{equation*}
  \item найдём функцию распределения
  \begin{equation*}
    F_{ \xi } \left( t \right) =
    P \left( \xi \leq t \right) =
    \int \limits_0^t 3e^{-3x} dx =
    \left. -e^{-3x} \right|_0^t =
    -e^{-3t} + 1 =
    1 - e^{-3t}, \,
    t > 0.
  \end{equation*}
  Учтём все случаи для параметра
  \begin{equation*}
    F_{ \xi } \left( t \right) =
    \begin{cases}
      1 - e^{-3t}, \qquad t > 0, \\
      0, \qquad t \leq 0;
    \end{cases}
  \end{equation*}
  \item найдём функцию распределения случайной величины $ \eta = e^{ \xi }$
  по определению
  \begin{equation*}
    F_{ \eta } \left( x \right) =
    P \left( \eta \leq x \right) =
    P \left( e^{ \xi } \leq x \right) =
    P \left( \xi \leq ln \, x \right) =
    F_{ \xi } \left( ln \, x \right) =
  \end{equation*}
  Из предыдущего пункта
  \begin{equation*}
    = \begin{cases}
      1 - e^{-3 ln \, x}, \qquad ln \, x > 0, \\
      0, \qquad ln \, x \leq 0
    \end{cases} =
    \begin{cases}
      1 - x^{-3}, \qquad x > 1, \\
      0, \qquad x \leq 1.
    \end{cases}
  \end{equation*}

  Продифференцируем функцию распределения
  \begin{equation*}
    p_{ \eta } \left( x \right) =
    \frac{dF_{ \eta } \left( x \right) }{dx} =
    \begin{cases}
      3x^{-4}, \qquad x > 1, \\
      0, \qquad x \leq 1.
    \end{cases}
  \end{equation*}
\end{enumerate}

\subsubsection*{1.2}

\textit{Задание.}
Пусть $ \xi $ и $ \eta $~---~независимые случайные величины,
причём $ \xi $ имеет дискретное распределение:
\begin{equation*}
  P \left( \xi = 0 \right) = 0.3, \,
  P \left( \xi = 1 \right) = 0.2, \,
  P \left( \xi = 2 \right) = 0.5,
\end{equation*}
а случайная величина $ \eta $ имеет показательное распределение с параметром
$ \lambda =
 2$.
Вычислите:
\begin{enumerate}[label=\alph*)]
  \item $M \xi \eta $;
  \item $M \left( 5 \xi^2 + 2 \eta^2 \right) $;
  \item $D \left( \sqrt{5} \xi + \sqrt{2} \eta \right) $;
  \item $D \xi \eta $.
\end{enumerate}
Запишите характеристические функции случайных величин $ \xi $ и $ \eta $
и характеристическую функцию случайной величины $ \xi + \eta $.

\textit{Решение.}
\begin{enumerate}[label=\alph*)]
  \item $ \xi, \, \eta $~---~независимы, поэтому
  \begin{equation*}
    M \left( \xi \eta \right) =
    M \xi \cdot M \eta =
    \left( 0 \cdot 0.3 + 1 \cdot 0.2 + 2 \cdot 0.5 \right) \cdot \frac{1}{2} =
    \left( 0.2 + 1 \right) \cdot \frac{1}{2} =
    1.2 \cdot \frac{1}{2}
    0.6;
  \end{equation*}
  \item вычислим
  \begin{equation*}
    M \left( 5 \xi^2 + 2 \eta^2 \right) =
    5M \xi^2 + 2M \eta^2 =
  \end{equation*}
  Посчитаем отдельно
  $M \xi^2 =
   0 \cdot 0.3 + 1 \cdot 0.2 + 4 \cdot 0.5 =
   0.2 + 2 =
   2.2$.
  Так как $D \eta = M \eta^2 - \left( M \eta \right)^2$, то
  \begin{equation*}
    M \eta^2 =
    D \eta + \left( M \eta \right)^2 =
    \frac{1}{4} + \frac{1}{4} =
    \frac{1}{2}.
  \end{equation*}
  Подставим полученные значения
  \begin{equation*}
    = 5 \cdot 2.2 + 2 \cdot \frac{1}{2} =
    11 + 1 =
    12;
  \end{equation*}
  \item из независимости случайных величин следует, что
  \begin{equation*}
    D \left( \sqrt{5} \xi + \sqrt{2} \eta \right) =
    5D \xi + 2D \eta =
  \end{equation*}
  Вычислим отдельно
  $D \xi =
   M \xi^2 - \left( M \xi \right)^2 =
   2.2 - \left( 1.2 \right)^2 =
   2.2 - 1.44 =
   0.76$.
  Подставим это значение
  \begin{equation*}
    = 5 \cdot 0.76 + 2 \cdot \frac{1}{4} =
    3.8 + 0.5 =
    4.3;
  \end{equation*}
  \item дисперсия произведения независимых случайных величин записывается по
  формуле
  \begin{equation*}
    D \left( \xi \eta \right) =
    D \xi \cdot D \eta + \left( M \xi \right)^2 \cdot D \eta +
    \left( M \eta \right)^2 \cdot D \xi =
  \end{equation*}
  Подставим значения
  \begin{equation*}
    = 0.76 \cdot 0.25 + 1.44 \cdot 0.25 + 0.25 \cdot 0.76 =
    0.19 + 0.36 + 0.19 =
    0.38 + 0.36 =
    0.54.
  \end{equation*}
\end{enumerate}

Запишем характеристическую функцию
\begin{equation*}
  \varphi_{ \xi } \left( t \right) =
  Me^{it \xi } =
  e^{it \cdot 0} \cdot 0.3 + e^{it \cdot 1} \cdot 0.2 +
  e^{it \cdot 2} \cdot 0.5 =
  0.3 + 0.2 e^{it} + 0.5 e^{2it}.
\end{equation*}

Запишем характеристическую функцию для $ \eta $ (непрерывной случайной величины)
\begin{equation*}
  \varphi_{ \eta } \left( t \right) =
  Me^{it \eta } =
  \int \limits_{-\infty }^{+\infty } e^{itx} p_{ \eta } \left( x \right) dx =
  \int \limits_0^{+\infty } e^{itx} \cdot 2 e^{-2x} dx =
  2 \int \limits_0^{+\infty } e^{-x \left( 2 - it \right) } dx =
\end{equation*}
Возьмём интеграл
\begin{equation*}
  = \left. -\frac{2}{2 - it} \right|_0^{+\infty } =
  \frac{2}{2 - it}.
\end{equation*}

Так как случайные величины независимы, то
\begin{equation*}
  \varphi_{ \xi + \eta } \left( t \right) =
  \varphi_{ \xi } \left( t \right) \cdot \varphi_{ \eta } \left( t \right) =
  \left( 0.3 + 0.2 e^{it} + 0.5 e^{2it} \right) \cdot \frac{2}{2 - it}.
\end{equation*}

\subsubsection*{1.3}

\textit{Задание.}
Плотность распределения случайного вектора $ \left( \xi_1, \xi_2 \right) $
равна:
\begin{equation*}
  p \left( x, y \right) =
  \begin{cases}
    \frac{1 + 9x^2 y^2}{8} \qquad at \, -1 \leq x, y \leq 1, \\
    0, \qquad otherwise.
  \end{cases}
\end{equation*}
Найдите:
\begin{enumerate}[label=\alph*)]
  \item $P \left( \xi_1 < \xi_2 \right) $;
  \item $M \xi_1 \xi_2$;
  \item плотности распределений случайных величин $ \xi_1$ и $ \xi_2$.
\end{enumerate}
Являются ли случайные величины $ \xi_1$ и $ \xi_2$ независимыми?

\textit{Решение.}
\begin{enumerate}[label=\alph*)]
  \item Найдём Вероятность
  \begin{equation*}
    P \left( \xi_1 < \xi_2 \right) =
    \iint \limits_{ \mathbb{R}^2}
      \mathbbm{1} \left\{ x < y \right\} \cdot p \left( x, y \right)
    dxdy =
    \iint \limits_{x < y} p \left( x, y \right) dxdy =
  \end{equation*}
  Подставим плотность распределения
  \begin{equation*}
    = \iint \limits_{x < y}
      \frac{1 + 9x^2 y^2}{8} \cdot
      \mathbbm{1} \left\{
        x \in \left[ -1, 1 \right], y \in \left[ -1, 1 \right]
      \right\}
    dxdy =
  \end{equation*}
  Поменяем пределы интегрирования за счёт индикатора
  \begin{equation*}
    = \frac{1}{8}
    \int \limits_{-1}^1 \int \limits_{-1}^y \left( 1 + 9x^2 y^2 \right) dxdy =
    \frac{1}{8} \int \limits_{-1}^1
      \left. \left( x + \frac{9x^3 y^2}{3} \right) \right|_{-1}^y
    dy =
  \end{equation*}
  Сократим константы
  \begin{equation*}
    = \frac{1}{8}
    \int \limits_{-1}^1 \left. \left(x + 3x^3 y^2 \right) \right|_{-1}^y dy =
    \frac{1}{8} \int \limits_{-1}^1 \left( y + 3y^3 y^2 + 1 + 3y^2 \right) dy =
  \end{equation*}
  Упростим
  \begin{equation*}
    = \frac{1}{8} \int \limits_{-1}^1 \left( y + 3y^5 + 1 + 3y^2 \right) dy =
    \frac{1}{8} \left.
      \left( \frac{y^2}{2} + \frac{3y^6}{6} + y + \frac{3y^3}{3} \right)
    \right|_{-1}^1 =
  \end{equation*}
  Сократим константы
  \begin{equation*}
    = \frac{1}{8} \left.
      \left( \frac{y^2}{2} + \frac{y^6}{2} + y + y^3 \right)
    \right|_{-1}^1 =
    \frac{1}{8} \left(
      \frac{1}{2} + \frac{1}{2} + 1 + 1 - \frac{1}{2} - \frac{1}{2} + 1 + 1
    \right) =
  \end{equation*}
  Некоторые слагаемые уничтожаются
  \begin{equation*}
    = \frac{1}{8} \cdot 4 =
    \frac{1}{2};
  \end{equation*}
  \item вычислим математическое ожидание
  \begin{equation*}
    M \xi_1 \xi_2 =
    \iint \limits_{ \mathbb{R}^2} xyp \left( x, y \right) dxdy =
  \end{equation*}
  Подставим вид $p \left( x, y \right) $.
  Получим
  \begin{equation*}
    = \int \limits_{-1}^1
      \int \limits_{-1}^1 \frac{1 + 9x^2 y^2}{8} \cdot dx
    dy =
    \frac{1}{8}
    \int \limits_{-1}^1 \int \limits_{-1}^1 \left( xy + 9x^3 y^3 \right) dxdy =
  \end{equation*}
  Возьмём внутренний интеграл
  \begin{equation*}
    = \frac{1}{8} \int \limits_{-1}^1
      \left. \left( \frac{x^2 y}{2} + \frac{9x^4 y^3}{4} \right) \right|_{-1}^1
    dy =
    \frac{1}{8} \int \limits_{-1}^1
      \left( \frac{y}{2} + \frac{9y^3}{4} - \frac{y}{2} - \frac{9y^3}{4} \right)
    dy =
    0;
  \end{equation*}
  \item находим плотность
  каждой из случайных величин как плотность компонент вектора
  \begin{equation*}
    p_{ \xi_1} \left( x \right) =
    \int \limits_{ \mathbb{R}} p \left( x, y \right) dy =
    \int \limits_{-1}^1 \frac{1 + 9x^2 y^2}{8} \cdot dy =
    \frac{1}{8} \int \limits_{-1}^1 \left( 1 + 9x^2 y^2 \right) dy =
  \end{equation*}
  Возьмём интеграл
  \begin{equation*}
    = \frac{1}{8} \left. \left( y + \frac{9x^2 y^3}{3} \right) \right|_{-1}^1 =
    \frac{1}{8} \left. \left( y + 3x^2 y^3 \right) \right|_{-1}^1 =
    \frac{1}{8} \left( 1 + 3x^2 + 1 + 3x^2 \right) =
  \end{equation*}
  Приведём подобные
  \begin{equation*}
    = \frac{1}{8} \left( 2 + 6x^2 \right) =
    \frac{1 + 3x^2}{4}.
  \end{equation*}

  Теперь ищем плотность $ \xi_2$ аналогичным образом
  \begin{equation*}
    p_{ \xi_2} \left( y \right) =
    \int \limits_{ \mathbb{R}} p \left( x, y \right) dx =
    \int \limits_{-1}^1 \frac{1 + 9x^2 y^2}{8} \cdot dx =
    \frac{1 + 3y^2}{4}.
  \end{equation*}
\end{enumerate}

Поскольку плотность
$p \left( x, y \right) \neq
 p_{ \xi_1} \left( x \right) \cdot p_{ \xi_2} \left( y \right) $,
то $ \xi_1$ и $ \xi_2$ не могут быть независимыми.

\subsubsection*{1.4}

\textit{Задание.}
Пусть $ \xi $ и
$ \eta $~---~независимые стандартные гауссовские случайные величины.
Докажите, что $ \xi + \eta $ и $ \xi - \eta $
являются независимыми случайными величинами.

\textit{Решение.}
$ \xi, \eta \sim N \left( 0, 1 \right) $.

Вектор $ \left( \xi, \eta \right) $~---~гауссовский вектор с вектором средних
$ \left( 0, 0 \right) $ и ковариационной матрицей
\begin{equation*}
  \begin{bmatrix}
    1 & 0 \\
    0 & 1
  \end{bmatrix}.
\end{equation*}

Сначала нужно показать,
что $ \left( \xi + \eta, \xi - \eta \right) $ является гауссовским вектором.
Берём какую-то комбинацию и группируем
\begin{equation*}
  \lambda_1 \left( \xi + \eta \right) + \lambda_2 \left( \xi - \eta \right) =
  \lambda_1 \xi + \lambda_1 \eta + \lambda_2 \xi - \lambda_2 \eta =
  \xi \left( \lambda_1 + \lambda_2 \right) +
  \eta \left( \lambda_1 - \lambda_2 \right).
\end{equation*}

Видим, что это линейная комбинация координат гауссовского вектора,
который имеет нормальное распределение.

Значит, вектор гауссовский.
Тогда независимость координат эквивалентна их некоррелируемости
$cov \left( \xi + \eta, \xi - \eta \right) =
 0$.

Воспользуемся линейностью ковариации
\begin{equation*}
  cov \left( \xi + \eta, \xi - \eta \right) =
  cov \left( \xi, \xi \right) - cov \left( \xi, \eta \right) +
  cov \left( \eta, \xi \right) - cov \left( \eta, \eta \right) =
  D \xi - D \eta =
  0,
\end{equation*}
так как $D \xi = D \eta = 1$.
Следовательно, случайные величины $ \xi + \eta $ и $ \xi - \eta $ независимы.
