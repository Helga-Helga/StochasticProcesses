\addcontentsline{toc}{chapter}{Занятие 12.
                                Регулярные и сингурярные стационарные последовательности.}
\chapter*{Занятие 12. Регулярные и сингурярные стационарные последовательности.}

\addcontentsline{toc}{section}{Контрольные вопросы и задания}
\section*{Контрольные вопросы и задания}

\subsubsection*{Приведите определение стационарной последовательности.}

Последовательность $ \left\{ \xi_n, n \in \mathbb{Z} \right\} $ называется стационарной
(в широком смысле), если
$M \xi_n = M \xi_0, n \in \mathbb{Z},
  cov \left( \xi_{n + m}, \xi_m \right) = cov \left( \xi_nn, \xi_0 \right), n, m \in \mathbb{Z}$.

\subsubsection*{Запишите спектральное изображение ковариационной функции стационарной
                последовательности}

Если $k$~---~ковариационная функция стационарной в широком смысле последовательности,
то сущетвует мера на
$ \left( \left[ -\pi, \pi \right], \beta \left( \left[ -\pi, \pi \right] \right) \right)$ такая, что
\begin{equation*}
  k \left( n \right) =
  \frac{1}{2 \pi } \int \limits_{-\pi }^{ \pi } e^{i \lambda n} \mu \left( d \lambda \right),
  n \in \mathbb{Z}.
\end{equation*}

\subsubsection*{Приведите определение подпространств будущего,
                связанных со стационарной последовательностью.}
$H_n^{ \xi } = \overline{LS \left\{ \xi_k, k \leq n \right\} }$~---~замыкание
в среднем квадратическом линейной оболочки
$ \left\{ \xi_k, k \leq n \right\} $~---~подпространство пространства
$L_2 \left( \Omega, \mathcal{F}, P \right) $.

\subsubsection*{Приведите определение регулярной и сингулярной последовательности.}

Стационарная последовательность называется сингулярной, если
\begin{equation*}
  H_{-\infty }^{ \xi } =
  \dotsc =
  H_0^{ \xi } =
  \dotsc,
\end{equation*}
то есть если они все просто совпадают между собой, и называется рещулярной,
если $H_{-\infty }^{ \xi } = \left\{ 0 \right\} $.

\subsubsection*{Сформулируйте задачу прогноза стационарной последовательности.}

Найти $ \hat{ \xi_n} = Q_{n_0} \xi_n, n_0 < n$, где $Q_n$~---~это проектор на $H_n$.

\subsubsection*{Сформулируйте критерий Колмогорова регулярности стационарной последовательности.}

Пусть $ \left\{ \xi_n \right\} $ имеет спектральную плотность $p$.
Тогда $ \left\{ \xi_n \right\} $~---~сингулярная или решулярная в зависимости от
\begin{equation*}
  \int \limits_{-\pi }^{ \pi } ln \, p \left( \lambda \right) d \lambda =
  \begin{cases}
    -\infty, \qquad singular, \\
    \in \mathbb{R}, \qquad regular.
  \end{cases}
\end{equation*}

\addcontentsline{toc}{section}{Аудиторные задачи}
\section*{Аудиторные задачи}

\addcontentsline{toc}{section}{Домашнее задание}
\section*{Домашнее задание}

\subsubsection*{12.10}

\textit{Задание.}
Стационарная последовательность имеет спектральную плотность
\begin{equation*}
  f \left( \lambda \right) =
  \frac{1}{2 \pi } \left| 4 + e^{i \lambda } \right|^2.
\end{equation*}
Докажите, что последовательность является регулярной, показав,
что её можно подать в виде одностороннего скользящего среднего.

\textit{Решение.}
\begin{equation*}
  f \left( \lambda \right) =
  \frac{1}{2 \pi } \left| 4 + e^{i \lambda } \right|^2 =
  \frac{1}{2 \pi } \left| 4e^{-i \lambda } + 1 \right|^2.
\end{equation*}

Представим $ \xi_n = \varepsilon_n + 4 \varepsilon_{n - 1}$,
где $ \left\{ \varepsilon_n \right\}_{n \in \mathbb{Z}}$~---~белый шум.

Последовательность одностороннего скользящего среднего имеет вид
\begin{equation*}
  \xi_n = \sum \limits_{k = 0}^{+\infty } c_k \varepsilon_{n - k}, \,
  \sum \limits_{k = 0}^{+\infty } \left| c_k \right|^2 < +\infty.
\end{equation*}

Тогда
\begin{equation*}
  \xi_n =
  \varepsilon_n + 4 \varepsilon_{n - 1} + \sum \limits_{k > 1} c_k \varepsilon_{n - k},
\end{equation*}
откуда $c_0 = 1, c_1 = 4, c_k = 0, k > 1$.

\subsubsection*{12.11}

\textit{Задание.}
Стационарная последовательность имеет спектральную плотность
\begin{equation*}
  f \left( \lambda \right) =
  \frac{1}{2 \pi } \left( 6 + \cos \lambda \right).
\end{equation*}
Докажите, что последовательность является регулярной, показав,
что её можно подать в виде одностороннего скользящего среднего.

\textit{Решение.}
\begin{equation*}
  f \left( \lambda \right) =
  \left| \varphi \left( \lambda \right) \right|^2 \cdot f_{ \varepsilon } \left( \lambda \right) =
  \frac{1}{2 \pi } \left( 6 + \cos \lambda \right),
\end{equation*}
где
\begin{equation*}
  f_{ \varepsilon } \left( \lambda \right) =
  \frac{1}{2 \pi }
\end{equation*}
это плотность белого шума.

Воспользуемся формулой Эйлера
\begin{equation*}
  6 + \cos \lambda =
  6 + \frac{e^{i \lambda } + e^{-i \lambda }}{2}.
\end{equation*}

Тогда
\begin{gather*}
  \left| \varphi \left( \lambda \right) \right|^2 =
  \left( c_1 + c_2 e^{i \lambda} \right) \left( c_1 + c_2 e^{-i \lambda } \right) =
  c_1^2 + c_1 c_2 e^{-i \lambda } + c_1 c_2 e^{i \lambda } + c_2^2 = \\
  = 6 + \frac{1}{2} \cdot e^{i \lambda } + \frac{1}{2} \cdot e^{-i \lambda }.
\end{gather*}

Находим коэффициенты из системы уравнений
\begin{equation*}
  \begin{cases}
    c_1^2 + c_2^2 = 6, \\
    c_1 c_2 = \frac{1}{2}.
  \end{cases}
\end{equation*}
Выразим из второго уравнения
\begin{equation*}
  c_1 =
  \frac{1}{2c_2}
\end{equation*}
и подставим в первое уравнение
\begin{equation*}
  \frac{1}{4c_2^2} + c_2^2 = 6.
\end{equation*}
Умножим уравнение на $4c_2^2$ и сделаем замену $c_2^2 = t$.
Получим квадратное уравнение $4t^2 - 24t + 1 = 0$.
Найдём дискриминант
\begin{equation*}
  D =
  576 - 16 = 560 = 16 \cdot 35.
\end{equation*}
Тогда корни имеют вид
\begin{equation*}
  t_{1, 2} =
  \frac{24 \pm 4 \sqrt{35}}{8} =
  \frac{6 \pm \sqrt{35}}{2},
\end{equation*}
откуда
\begin{equation*}
  c_2 = \frac{ \left( 6 + \sqrt{35} \right)^2}{4}, \,
  c_1 = \frac{2}{ \left( 6 + \sqrt{35} \right)^2}.
\end{equation*}
Тогда
\begin{equation*}
  \varphi \left( \lambda \right) =
  c_1 + c_2 e^{-i \lambda } =
  \sum \limits_{k = 0}^{+\infty } a_k e^{-ik \lambda },
\end{equation*}
откуда $a_0 = c_1, a_1 = c_2, a_k = 0, k > 1$.

\subsubsection*{12.12}

\textit{Задание.}
Спектральная плотность стационарной последовательности
\begin{equation*}
  \left\{ \xi_n \right\}_{n \in \mathbb{Z}}
\end{equation*}
равна $f \left( \lambda \right) = e^{-\frac{1}{ \left| \lambda \right| }}$.
Выясните, является ли последовательность регулярной, сингулярной.

\textit{Решение.}
Согласно критерию Колмогорова,
стационарная последовательность со спектральной плотностью $f \left( \lambda \right) $
является регулярной тогда и только тогда, когда
\begin{equation*}
  \int \limits_{-\pi }^{ \pi } ln f \left( \lambda \right) d \lambda >
  -\infty.
\end{equation*}
Проверим выполнимость этого условия для заданной спектральной плотности.
Имеем:
\begin{equation*}
  \int \limits_{-\pi }^{ \pi } ln e^{-\frac{1}{ \left| \lambda \right| }} d \lambda =
  -\int \limits_{-\pi }^{ \pi } \frac{1}{ \left| \lambda \right| } d \lambda =
  \int \limits_{-\pi }^0 \frac{1}{ \lambda } d \lambda -
  \int \limits_0^{ \pi } \frac{1}{ \lambda } d \lambda =
  -2 \int \limits_0^{ \pi } \frac{1}{ \lambda } d \lambda =
  -2 \left. ln \lambda \right|_0^{ \pi } =
\end{equation*}
Подставим пределы интегрирования
\begin{equation*}
  = -2 ln \pi + 2 ln 0 =
  -2 ln \pi - 2 \cdot \infty =
  -\infty.
\end{equation*}

Из бесконечности этого интеграла следует,
что последовательность
\begin{equation*}
  \left\{ \xi_n \right\}_{n \in \mathbb{Z}}
\end{equation*}
является сингулярной.

\subsubsection*{12.13}

\textit{Задание.}
Докажите, что последовательность со спектральной плотностью
$f \left( \lambda \right) =
  \mathbbm{1}_{ \left[ 0, \pi \right] } \left( \lambda \right) $
является сингулярной.

\textit{Решение.}
Согласно критерию Колмогорова,
стационарная последовательность со спектральной плотностью $f \left( \lambda \right) $
является сингулярной тогда и только тогда, когда
\begin{equation*}
  \int \limits_{-\pi }^{ \pi } ln f \left( \lambda \right) d \lambda =
  -\infty.
\end{equation*}

Проверим выполнимость этого условия для заданной спектральной плотности.
Имеем
\begin{equation*}
  \int \limits_{-\pi }^{ \pi }
    ln \mathbbm{1}_{ \left[ 0, \pi \right] } \left( \lambda \right) d \lambda =
  \int \limits_{-\pi }^0 ln \mathbbm{1}_{ \left[ 0, \pi \right] } \left( \lambda \right) d \lambda +
  \int \limits_0^{ \pi } ln \mathbbm{1}_{ \left[ 0, \pi \right] } \left( \lambda \right) d \lambda =
  \int \limits_{-\pi }^0 ln 0 d \lambda + \int \limits_0^{ \pi } ln 1 d \lambda =
\end{equation*}
Второй интеграл равен нулю
\begin{equation*}
  = -\infty.
\end{equation*}

Из бесконечности этого интеграла следует,
что последовательность $ \left\{ \xi_n \right\}_{n \in \mathbb{Z}}$ сингулярна.

\subsubsection*{12.14}

\textit{Задание.}
Пусть $ \left\{ \xi_n \right\}_{n \in \mathbb{Z}}$ и
$ \left\{ \eta_n \right\}_{n \in \mathbb{Z}}$~---~ортогональные регулярные последовательности.
Докажите, что сумма $ \left\{ \xi_n + \eta_n \right\}_{n \in \mathbb{Z}}$
является регулярной последовательностью.

\textit{Решение.}
Если последовательности $ \left\{ \xi_n \right\}_{n \in \mathbb{Z}}$ и
$ \left\{ \eta_n \right\}_{n \in \mathbb{Z}}$ регулярны,
то их можно представить в виде одностороннего скользящего среднего, то есть
\begin{equation*}
  \xi_n = \sum \limits_{k = 0}^{+\infty } a_k \varepsilon_{n - k}, \,
  \eta_n = \sum \limits_{k = 0}^{+\infty } b_k \varepsilon_{n - k},
\end{equation*}
где
\begin{equation*}
  \sum \limits_{k = 0}^{+\infty } \left| a_k \right|^2 < +\infty, \,
  \sum \limits_{k = 0}^{+\infty } \left| b_k \right|^2 < +\infty.
\end{equation*}
Тогда
\begin{equation*}
  \xi_n + \eta_n =
  \sum \limits_{k = 0}^{+\infty } a_k \varepsilon_{n - k} +
  \sum \limits_{k = 0}^{+\infty } b_k \varepsilon_{n - k} =
  \sum \limits_{k = 0}^{+\infty } \left( a_k + b_k \right) \varepsilon_{n - k} =
  \sum \limits_{k = 0}^{+\infty } c_k \varepsilon_{n - k},
\end{equation*}
поэтому $ \left\{ \xi_n + \eta_n \right\}_{n \in \mathbb{Z}}$~---~регулярная последовательность.
