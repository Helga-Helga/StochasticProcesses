\addcontentsline{toc}{chapter}{Занятие 12.
                                Регулярные и сингурярные стационарные последовательности.}
\chapter*{Занятие 12. Регулярные и сингурярные стационарные последовательности.}

\addcontentsline{toc}{section}{Контрольные вопросы и задания}
\section*{Контрольные вопросы и задания}

\subsubsection*{Приведите определение стационарной последовательноси.}

Последовательность $ \left\{ \xi_n, n \in \mathbb{Z} \right\} $ называется стационарной
(в широком смысле), если
$M \xi_n = M \xi_0, n \in \mathbb{Z},
  cov \left( \xi_{n + m}, \xi_m \right) = cov \left( \xi_nn, \xi_0 \right), n, m \in \mathbb{Z}$.

\subsubsection*{Запишите спектральное изображение ковариационной функции стационарной
                последовательности}

Если $k$~---~ковариационная функция стационарной в широком смысле последовательности,
то сущетвует мера на
$ \left( \left[ -\pi, \pi \right], \beta \left( \left[ -\pi, \pi \right] \right) \right)$ такая, что
\begin{equation*}
  k \left( n \right) =
  \frac{1}{2 \pi } \int \limits_{-\pi }^{ \pi } e^{i \lambda n} \mu \left( d \lambda \right),
  n \in \mathbb{Z}.
\end{equation*}

\subsubsection*{Приведите определение подпространств будущего,
                связанных со стационарной последовательностью.}
$H_n^{ \xi } = \overline{LS \left\{ \xi_k, k \leq n \right\} }$~---~замыкание
в среднем квадратическом линейной оболочки
$ \left\{ \xi_k, k \leq n \right\} $~---~подпространство пространства
$L_2 \left( \Omega, \mathcal{F}, P \right) $.

\subsubsection*{Приведите определение регулярной и сингулярной последовательности.}

Стационарная последовательность называется сингулярной, если
\begin{equation*}
  H_{-\infty }^{ \xi } =
  \dotsc =
  H_0^{ \xi } =
  \dotsc,
\end{equation*}
то есть если они все просто совпадают между собой, и называется рещулярной,
если $H_{-\infty }^{ \xi } = \left\{ 0 \right\} $.

\subsubsection*{Сформулируйте задачу прогноза стационарной последовательности.}

Найти $ \hat{ \xi_n} = Q_{n_0} \xi_n, n_0 < n$, где $Q_n$~---~это проектор на $H_n$.

\subsubsection*{Сформулируйте критерий Колмогорова регулярности стационарной последовательности.}

Пусть $ \left\{ \xi_n \right\} $ имеет спектральную плотность $p$.
Тогда $ \left\{ \xi_n \right\} $~---~сингулярная или решулярная в зависимости от
\begin{equation*}
  \int \limits_{-\pi }^{ \pi } ln \, p \left( \lambda \right) d \lambda =
  \begin{cases}
    -\infty, \qquad singular, \\
    \in \mathbb{R}, \qquad regular.
  \end{cases}
\end{equation*}

\addcontentsline{toc}{section}{Аудиторные задачи}
\section*{Аудиторные задачи}

\addcontentsline{toc}{section}{Домашнее задание}
\section*{Домашнее задание}

\subsubsection*{12.10}

\textit{Задание.}
Стационарная последовательность имеет спектральную плотность
\begin{equation*}
  f \left( \lambda \right) =
  \frac{1}{2 \pi } \left| 4 + e^{i \lambda } \right|^2.
\end{equation*}
Докажите, что последовательность является регулярной, показав,
что её можно подать в виде одностороннего скользящего среднего.

\textit{Решение.}
\begin{equation*}
  f \left( \lambda \right) =
  \frac{1}{2 \pi } \left| 4 + e^{i \lambda } \right|^2 =
  \frac{1}{2 \pi } \left| 4e^{-i \lambda } + 1 \right|^2.
\end{equation*}

Представим $ \xi_n = \varepsilon_n + 4 \varepsilon_{n - 1}$,
где $ \left\{ \varepsilon_n \right\}_{n \in \mathbb{Z}}$~---~белый шум.

Последовательность одностороннего скользящего среднего имеет вид
\begin{equation*}
  \xi_n = \sum \limits_{k = 0}^{+\infty } c_k \varepsilon_{n - k}, \,
  \sum \limits_{k = 0}^{+\infty } \left| c_k \right|^2 < +\infty.
\end{equation*}

Тогда
\begin{equation*}
  \xi_n =
  \varepsilon_n + 4 \varepsilon_{n - 1} + \sum \limits_{k > 1} c_k \varepsilon_{n - k},
\end{equation*}
откуда $c_0 = 1, c_1 = 4, c_k = 0, k > 1$.
