\addcontentsline{toc}{chapter}{Занятие 5. Винеровский процесс}
\chapter*{Занятие 5. Винеровский процесс}

\addcontentsline{toc}{section}{Контрольные вопросы и задания}
\section*{Контрольные вопросы и задания}

\subsubsection*{Приведите определение винеровского процесса.}

$ \left\{ w \left( t \right), \, t \geq 0 \right\} $ обладает рядом свойств:
\begin{enumerate}
  \item $w \left( 0 \right) = 0$;
  \item однородные приращения.
  Рассмотрим приращение винеровского процесса на $t$.
  Тогда
  $w \left( s + t \right) - w \left( s \right) \overset{def}{=}
    w \left( t \right) \sim
    N \left( 0, t \right) $;
  \item независимые приращения.
  Выберем $0 < t_1 < t_2 < \dotsc < t_n$.
  Тогда
  $w \left( t_1 \right), \,
    w \left( t_2 \right) - w \left( t_1 \right),
    \dotsc,
    w \left( t_n \right) - w \left( t_{n - 1} \right) $ ---
  независимые в совокупности случайные величины.
\end{enumerate}

\subsubsection*{Запишите плотность винеровского процесса.}

Напишем плотность распределения вектора
$ \left( w \left( t_1 \right), \dotsc, w \left( t_n \right) \right) =
  \vec{ \xi }$.

Будем использовать матрицу
$$A =
  \begin{bmatrix}
    1 & 0 & 0 & 0 & \dotsc & 0 \\
    1 & 1 & 0 & 0 & \dotsc & 0 \\
    1 & 1 & 1 & 0 & \dotsc & 0 \\
    \dotsc \\
    1 & 1 & 1 & 1 & \dotsc & 1
  \end{bmatrix}.$$

Таким образом $ \vec{ \xi }$ имеет плотность
$$q \left( A^{-1} \vec{u} \right) =
  \prod \limits_{j = 0}^{n - 1}
    \frac{1}{ \sqrt{2 \pi \left( t_{j + 1} - t_j \right) }} \cdot e^{-\frac{u_{j + 1} - u_j}{2t_{j + 1} - t_j}}.$$
В этой плотности считаем, что $t_0 = 0, \, u_0 = 0$.

\subsubsection*{Запишите ковариационную функцию винеровского процесса.}

$K \left( t, s \right) = \min \left( s, t \right) $.

\addcontentsline{toc}{section}{Аудиторные задачи}
\section*{Аудиторные задачи}

\subsubsection*{5.2}

\textit{Задание.}
Пусть $ \left\{ W \left( t \right), \, t \geq 0 \right\} $ --- винеровский процесс.
Докажите, что
$M \left( W \left( t \right) - W \left( s \right) \right)^{2n + 1} = 0, \,
  M \left( W \left( t \right) - W \left( s \right) \right)^{2n} =
  \left( 2n - 1 \right)!! \left( t - s \right)^n$.

\textit{Решение.}
$ \xi = W \left( t \right) - W \left( s \right) \overset{def}{=} W \left( t - s \right) $.
Значит, $ \xi \sim N \left( 0, t - s \right) $, где $t - s = \sigma^2$.
Знаем, что $M \xi^{2n + 1} = 0, \, M \xi^{2n} = \left( 2n + 1 \right)!! \sigma^{2n}$.

\subsubsection*{5.3}

\textit{Задание.}
Пусть $ \left\{ W \left( t \right), \, t \geq 0 \right\} $ --- винеровский процесс.
Вычислите:
\begin{enumerate}[label=\alph*)]
  \item $M \left[ \left( W \left( 5 \right) - 2W \left( 1 \right) + 2 \right)^3 \right] $;
  \item характеристическую функцию случайной величины $W \left( 2 \right) + 2W \left( 1 \right) $;
  \item $M \left[ \sin \left( 2W \left( 1 \right) + W \left( 2 \right) \right) \right] $;
  \item $M \left[ \cos \left( 2W \left( 1 \right) + W \left( 2 \right) \right) \right] $.
\end{enumerate}

\textit{Решение.}
\begin{enumerate}[label=\alph*)]
  \item $W \left( 5 \right) - 2W \left( 1 \right) + 2 = \xi \sim N \left( 2, 5 \right) $.
  Найдём дисперсию
  $$D \xi =
    cov \left( \xi, \xi \right) =
    cov \left[ W \left( 5 \right) - 2W \left( 1 \right) + 2, \,
      W \left( 5 \right) - 2W \left( 1 \right) + 2 \right] =$$
  Воспользуемся линейностью
  $$= K \left( 5, 5 \right) - 2K \left( 5, 1 \right) - 2K \left( 5, 1 \right) +
    4K \left( 1, 1 \right) =
    5 - 2 - 2 + 4 =
    5.$$
  Нужно $M \xi^3 = M \left[ \left( \xi - 2 \right) + 2 \right]^3 $.
  Раскрываем скобки
  $$M \xi^3 =
    M \left( \xi - 2 \right)^3 + 6M \left( \xi - 2 \right)^3 + 12M \left( \xi - 2 \right) + 8.$$
  По предыдущей задаче $M \xi^3 = 0 + 6 \cdot 5 + 12 \cdot 0 + 8 = 38$.

  Величины $W \left( 5 \right) $ и $W \left( 1 \right) $ --- зависимы,
  а приращения в винеровском процессе --- независимы
  $$D \left[ W \left( 5 \right) - 2W \left( 1 \right) \right] =
    D \left\{
      \left[ W \left( 5 \right) - W \left( 1 \right) \right] + \left[ -W \left( 1 \right) \right]
    \right\}.$$
  Дисперсия первого слагаемого равна 4, а второго --- 1.
  Слагаемые независимы $D \left[ W \left( 5 \right) - 2W \left( 1 \right) \right] = 5$.
\end{enumerate}
