\addcontentsline{toc}{chapter}{Занятие 5. Винеровский процесс}
\chapter*{Занятие 5. Винеровский процесс}

\addcontentsline{toc}{section}{Контрольные вопросы и задания}
\section*{Контрольные вопросы и задания}

\subsubsection*{Приведите определение винеровского процесса.}

$ \left\{ w \left( t \right), \, t \geq 0 \right\} $ --- винеровский процесс,
если обладает рядом свойств:
\begin{enumerate}
  \item $w \left( 0 \right) = 0$;
  \item однородные приращения.
  Рассмотрим приращение винеровского процесса на $t$.
  Тогда
  $w \left( s + t \right) - w \left( s \right) \overset{def}{=}
    w \left( t \right) \sim
    N \left( 0, t \right) $, то есть распределение процесса зависит только от длины отрезка;
  \item независимые приращения на непересекающихся отрезках.
  Выберем $0 < t_1 < t_2 < \dotsc < t_n$.
  Тогда
  $w \left( t_1 \right), \,
    w \left( t_2 \right) - w \left( t_1 \right),
    \dotsc,
    w \left( t_n \right) - w \left( t_{n - 1} \right) $ ---
  независимые в совокупности случайные величины.
\end{enumerate}

\subsubsection*{Запишите плотность винеровского процесса.}

Напишем плотность распределения вектора
$ \left( w \left( t_1 \right), \dotsc, w \left( t_n \right) \right) =
  \vec{ \xi }$.

Будем использовать матрицу
$$A =
  \begin{bmatrix}
    1 & 0 & 0 & 0 & \dotsc & 0 \\
    1 & 1 & 0 & 0 & \dotsc & 0 \\
    1 & 1 & 1 & 0 & \dotsc & 0 \\
    \dotsc \\
    1 & 1 & 1 & 1 & \dotsc & 1
  \end{bmatrix}.$$

Таким образом $ \vec{ \xi }$ имеет плотность
$$q \left( A^{-1} \vec{u} \right) =
  \prod \limits_{j = 0}^{n - 1}
    \frac{1}{ \sqrt{2 \pi \left( t_{j + 1} - t_j \right) }} \cdot e^{-\frac{u_{j + 1} - u_j}{2t_{j + 1} - t_j}}.$$
В этой плотности считаем, что $t_0 = 0, \, u_0 = 0$.

\subsubsection*{Запишите ковариационную функцию винеровского процесса.}

Произведение математических ожиданий --- это 0, потому
$$K \left( t, s \right) =
  Mw \left( s \right) w \left( t \right) =$$
Используем независимость приращений
$$= M \left\{
    w \left( s \right) \cdot
    \left[ w \left( s \right) + \left( w \left( t \right) - w \left( s \right) \right) \right]
  \right\} =$$
Раскрываем скобки
$$= M \left\{
    w^2 \left(s \right) + Mw \left( s \right) \left[ w \left( t \right) - w \left( s \right) \right]
  \right\} =$$
Первое слагаемое равно $s$, а второе --- нулю,
так как это независимые центрированные случайные величины (математическое произведения ---
это произведение математических ожиданий, а они равны нулю)
$$= s, \,
  s < t.$$

$K \left( t, s \right) = \min \left( s, t \right) $.

\addcontentsline{toc}{section}{Аудиторные задачи}
\section*{Аудиторные задачи}

\subsubsection*{5.2}

\textit{Задание.}
Пусть $ \left\{ W \left( t \right), \, t \geq 0 \right\} $ --- винеровский процесс.
Докажите, что
$M \left( W \left( t \right) - W \left( s \right) \right)^{2n + 1} = 0, \,
  M \left( W \left( t \right) - W \left( s \right) \right)^{2n} =
  \left( 2n - 1 \right)!! \left( t - s \right)^n$.

\textit{Решение.}
Приращение гауссовское.
Обозначим
$$ \xi =
  W \left( t \right) - W \left( s \right) \overset{def}{=}
  W \left( t - s \right).$$
Значит, $ \xi \sim N \left( 0, t - s \right) $, где $t - s = \sigma^2$.
Нужны формулы для моментов центрированной гауссовской случайной величины, то есть
Знаем, что $M \xi^{2n + 1} = 0, \, M \xi^{2n} = \left( 2n + 1 \right)!! \sigma^{2n}$.

\subsubsection*{5.3}

\textit{Задание.}
Пусть $ \left\{ W \left( t \right), \, t \geq 0 \right\} $ --- винеровский процесс.
Вычислите:
\begin{enumerate}[label=\alph*)]
  \item $M \left[ \left( W \left( 5 \right) - 2W \left( 1 \right) + 2 \right)^3 \right] $;
  \item характеристическую функцию случайной величины $W \left( 2 \right) + 2W \left( 1 \right) $;
  \item $M \left[ \sin \left( 2W \left( 1 \right) + W \left( 2 \right) \right) \right] $;
  \item $M \left[ \cos \left( 2W \left( 1 \right) + W \left( 2 \right) \right) \right] $.
\end{enumerate}

\textit{Решение.}
Есть винеровский процесс.
\begin{enumerate}[label=\alph*)]
  \item $W \left( 5 \right) - 2W \left( 1 \right) + 2 = \xi \sim N \left( 2, 5 \right) $,
  потому что это линейная комбинация элементов гауссовского вектора.
  Найдём дисперсию.
  Константа на неё не влияет
  $$D \xi =
    D \left[ W \left( 5 \right) - 2W \left( 1 \right) \right] =
    cov \left( \xi, \xi \right) =$$
  Подставим выражения для случайной величины
  $$= cov \left[ W \left( 5 \right) - 2W \left( 1 \right) + 2, \,
      W \left( 5 \right) - 2W \left( 1 \right) + 2 \right] =$$
  Воспользуемся линейностью
  $$= K \left( 5, 5 \right) - 2K \left( 5, 1 \right) - 2K \left( 5, 1 \right) +
    4K \left( 1, 1 \right) =
    5 - 2 - 2 + 4 =
    5.$$
  Нужно найти третий момент.
  $ \xi $ не центрирована.
  Нужно её центрировать $M \xi^3 = M \left[ \left( \xi - 2 \right) + 2 \right]^3 $.
  Раскрываем скобки
  $$M \xi^3 =
    M \left( \xi - 2 \right)^3 + 6M \left( \xi - 2 \right)^3 + 12M \left( \xi - 2 \right) + 8.$$
  По предыдущей задаче первое слагаемое --- 0, так как величина центрирована, второй момент --- 5,
  так как это дисперсия, первый момент --- 0.
  Тогда $M \xi^3 = 0 + 6 \cdot 5 + 12 \cdot 0 + 8 = 38$.

  Величины $W \left( 5 \right) $ и $W \left( 1 \right) $ --- зависимы,
  а приращения в винеровском процессе --- независимы, потому имеем сумму дисперсий
  $$D \left[ W \left( 5 \right) - 2W \left( 1 \right) \right] =
    D \left\{
      \left[ W \left( 5 \right) - W \left( 1 \right) \right] + \left[ -W \left( 1 \right) \right]
    \right\}.$$
  Дисперсия первого слагаемого равна 4, а второго --- 1.
  Слагаемые независимы $D \left[ W \left( 5 \right) - 2W \left( 1 \right) \right] = 5$;
  \item нужно найти характеристическую функцию $W \left( 2 \right) + 2W \left( 1 \right) $.

  Математическое ожидание такой величины равно нулю, а дисперсия
  $D \left[ W \left( 2 \right) + 2W \left( 1 \right) \right] =
    D \left\{
      \left[ W \left( 2 \right) - W \left( 1 \right) \right] + 3W \left( 1 \right)
    \right\}$.
  Это независимые величины, поэтому
  $D \left\{
      \left[ W \left( 2 \right) - W \left( 1 \right) \right] + 3W \left( 1 \right)
    \right\} =
    1 + 9 = 10$.
  Значит, получается
  $ \varphi_{W \left( 2 \right) + 2W \left( 1 \right) } \left( \lambda \right) =
    \varphi_{N \left( 0, 10 \right) } \left( \lambda \right) =
    e^{-\frac{10 \lambda^2}{2}}$;
  \item $M \left[ \sin \left( 2W \left( 1 \right) + W \left( 2 \right) \right) \right] = 0$.

  Характеристическая функция случайной величины --- это
  $$ \varphi_{ \xi } \left( \lambda \right) = Me^{i \lambda \xi } =
    M \cos \lambda \xi + iM \sin \lambda \xi, \,
    \lambda = 1;$$
  \item $M \left[ \cos \left( 2W \left( 1 \right) + W \left( 2 \right) \right) \right] = e^{-5}$.
\end{enumerate}

\addcontentsline{toc}{section}{Домашнее задание}
\section*{Домашнее задание}

\subsubsection*{5.12}

\textit{Задание.}
Пусть $ \left\{ W \left( t \right), \, t \geq 0 \right\} $ --- винеровский процесс.
Вычислите:
\begin{enumerate}[label=\alph*)]
  \item $M \left[
      \left( W \left( 4 \right) - 2W \left( 1 \right) + 2W \left( 2 \right) \right)^2 \right] $;
  \item $M \left[ \left( W \left( 1 \right) + 2W \left( 2 \right) + 1 \right)^3 \right] $;
  \item $M \left[ e^{W \left( 3 \right) - 2W \left( 2 \right) } \right] $;
  \item характеристическую функцию случайной величины $W \left( 1 \right) + 2W \left( 2 \right) + 1$.
\end{enumerate}

\textit{Решение.}
\begin{enumerate}[label=\alph*)]
  \item $W \left( 4 \right) - 2W \left( 1 \right) + W \left( 2 \right) =
    \xi \sim
    N \left( 0, 6 \right) $,
  потому что это линейная комбинация элементов гауссовского вектора.
  Найдём дисперссию
  $$D \xi =
    D \left[ W \left( 4 \right) - 2W \left( 1 \right) + W \left( 2 \right) \right] =
    cov \left( \xi, \xi \right) =$$
  Подставим выражения для случайной величины
  $$= cov \left[
      W \left( 4 \right) - 2W \left( 1 \right) + W \left( 2 \right),
      W \left( 4 \right) - 2W \left( 1 \right) + W \left( 2 \right) \right] =$$
  Воспользуемся линейностью
  \begin{gather*}
    = K \left( 4, 4 \right) - 2K \left( 4, 1 \right) + K \left( 4, 2 \right) -
    2K \left( 1, 4 \right) + 4K \left( 1, 1 \right) - 2K \left( 1, 2 \right) + \\
    + K \left( 2, 4 \right) - 2K \left( 2, 1 \right) + K \left( 2, 2 \right) = \\
    = 4 - 2 \cdot 1 + 2 - 2 \cdot 1 + 4 \cdot 1 - 2 \cdot 1 + 2 - 2 \cdot 1 + 2 =
    4 + 4 - 2 =
    6.
  \end{gather*}

  Нужно найти второй момент.
  $ \xi $ центрирована $M \xi^2 = D \xi = 6$;
  \item $W \left( 1 \right) + 2W \left( 2 \right) + 1 =
    \xi \sim
    N \left( 1, 5 \right) $,
  потому ято это линейная комбинация элементов гауссовского вектора.
  Найдём дисперсию.
  Константа на неё не влияет
  $D \xi =
    D \left[ W \left( 1 \right) - 2W \left( 2 \right) \right] =
    cov \left( \xi, \xi \right) $.
  Подставим выражения для случайной величины
  $$cov \left( \xi, \xi \right) =
    cov \left[
      W \left( 1 \right) + 2W \left( 2 \right) + 1,
      W \left( 1 \right) + 2W \left( 2 \right) + 1 \right] =$$
  Воспользуемся линейностью
  $$= \left( 1, 1 \right) + 2K \left( 1, 2 \right) + 2K \left( 2, 1 \right) +
    4K \left( 2, 2 \right) =
    1 + 2 + 2 + 8 =
    13.$$

  Нужно найти третий момент.
  $ \xi $ не центрирована.
  Нужно её центрировать $M \xi^3 = M \left[ \left( \xi - 1 \right) + 1 \right]^3$.
  Раскрываем скобки
  $$M \left[ \left( \xi - 1 \right) + 1 \right]^3 =
    M \left( \xi - 1 \right)^3 + 3M \left( \xi - 1 \right) + 3M \left( \xi - 1 \right) + 1 =$$
  По задаче 5.2 первое слагаемое --- 0, так как величина центрирована, второй момент --- 5,
  так как это дисперсия, первый момент --- ноль.
  Тогда
  $$= 0 + 3 \cdot 13 + 3 \cdot 0 + 1 = 38 + 1 = 40;$$
  \item $W \left( 3 \right) - 2W \left( 2 \right) = \xi \sim N \left( 0, 3 \right) $,
  потому что это линейная комбинация элементов гауссовского вектора.
  Найдём дисперсию
  $$D \xi =
    D \left[ W \left( 3 \right) - 2W \left( 2 \right) \right] =
    cov \left( \xi, \xi \right) =$$
  Подставим выражение для случайной величины
  $$= cov \left[
      W \left( 3 \right) - 2W \left( 2 \right), W \left( 3 \right) - 2W \left( 2 \right) \right] =$$
  Воспользуемся линейностью
  $$= K \left( 3, 3 \right) - 2K \left( 3, 2 \right) - 2K \left( 2, 3 \right) +
    4K \left( 2, 2 \right) =
    3 - 2 \cdot 2 - 2 \cdot 2 + 3 \cdot 2 =
    3.$$
  Нужно найти
  $$Me^{ \xi } =
    \int \limits_{ \mathbb{R}}
      \frac{1}{ \sqrt{6 \pi }} \cdot e^x \cdot p_{ \xi } \left( x \right) dx =
    \int \limits_{ \mathbb{R}}
      \frac{1}{6 \pi } \cdot e^x \cdot \frac{1}{6 \pi } \cdot e^{- \frac{x^2}{2 \cdot 9}} dx =
    \int \limits_{ \mathbb{R}} \frac{1}{ \sqrt{6 \pi }} \cdot e^{x - \frac{x^2}{18}} dx.$$
  Выделим полный квадрат в степени экспоненты
  $$ \frac{x^2}{18} - x =
    \frac{x^2}{ \left( 3 \sqrt{2} \right)^2} -
    2 \cdot \frac{1}{2} \cdot x \cdot \frac{1}{3 \sqrt{2}} \cdot 2 \sqrt{2} +
    \left( \frac{1}{2} \right)^2 \cdot \left( 2 \sqrt{2} \right)^2 =$$
  Три первых слагаемых образуют полный квадрат
  $$= \left( \frac{x}{3 \sqrt{2}} - \frac{3 \sqrt{2}}{2} \right)^2 + \frac{ 9 \cdot 2}{4} =
    \left( \frac{x}{3 \sqrt{2}} - \frac{3}{ \sqrt{2}} \right)^2 + \frac{9}{2} =
    \frac{ \left( x - 9 \right)^2}{2 \cdot 3} + \frac{9}{2}.$$
  Подставим полученное выражение в экспоненту
  $$ \int \limits_{ \mathbb{R}} \frac{1}{ \sqrt{6 \pi }} \cdot e^{x - \frac{x^2}{18}} dx =
    e^{- \frac{9}{2}}
    \int \limits_{ \mathbb{R}}
      \frac{1}{ \sqrt{6 \pi }} \cdot e^{- \frac{ \left( x - 9 \right)^2}{2 \cdot 9}} dx =$$
  Умножим и поделим на $ \sqrt{3}$ чтобы получить гауссовскую плотность
  $$= \sqrt{3} e^{- \frac{9}{2}}
    \int \limits_{ \mathbb{R}}
      \frac{1}{ \sqrt{2 \pi \cdot 9}} \cdot e^{-\frac{ \left( x - 9 \right)^2}{2 \cdot 9}} dx =$$
  Подинтергальная функция --- плотность нормального распределения,
  потому такой интеграл равен единице
  $$= \sqrt{3} e^{- \frac{9}{2}}.$$
\end{enumerate}
