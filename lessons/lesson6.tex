не \addcontentsline{toc}{chapter}{Занятие 6. Стохастическая непрерывность случайного процесса.
                                Существование непрерывной модификации}
\chapter*{Занятие 6. Стохастическая непрерывность случайного процесса.
          Существование непрерывной модификации}

\addcontentsline{toc}{section}{Контрольные вопросы и задания}
\section*{Контрольные вопросы и задания}

\subsubsection*{Приведите опредедение стохастически непрерывного процесса.}

Стохастически непрерывный процесс:
$ \xi \left( t \right) \overset{P} \xi \left( t_0 \right), \, t \to t_0$.

Это означает, что
$ \forall \varepsilon > 0 \qquad
  P \left( \left| \xi \left( t \right) - \xi \left( t_0 \right) \right| > \varepsilon \right) \to 0,
  \, t \to t_0$.

\subsubsection*{Сформулируйте достаточное условие существования непрерывной модификации случайного
                процесса.}

Пусть $ \xi \left( t \right), \, t \in \left[ 0, 1 \right] $ удовлетворяет условию
$$ \exists \alpha, \beta, C > 0: \qquad
  \forall t_1, t_2 \in \left[ 0, 1 \right] \qquad
  M \left| \xi \left( t_1 \right) =
  \xi \left( t_2 \right) \right|^{ \alpha } \leq C \left| t_1 - t_2 \right|^{1 + \beta }.$$
Тогда $ \xi $ имеет непрерывную модификацию.

\addcontentsline{toc}{section}{Аудиторные задачи}
\section*{Аудиторные задачи}

\subsubsection*{6.7}

\textit{Задание.}
Пусть $ \left\{ \xi \left( t \right), \, t \in \left[ a, b \right] \right\} $ ---
стохастически непрерывный процесс, а $f$ --- неслучайная функция,
определённая на $ \left[ a, b \right] $.
Докажите, что случайный процесс
$ \eta \left( t \right) =
  \xi \left( t \right) + f \left( t \right), \, t \in \left[ a, b \right] $
является стохастически непрерывным в тех и только тех точках отрезка $ \left[ a, b \right] $,
где является непрерывной функция $f$.

\textit{Решение.}
Нужно доказывать в обе стороны.

Сначала предположим, что $f$ --- непрерывная.
Пусть $f$ --- непрерывная в точке $t_0$.
Будем сейчас проверять, что сумма стохастически непрерывна.

Если $ \xi \left( t \right) $ --- стохастически непрерывна, то
$$ \xi \left( t \right) \overset{P}{ \to }
  \xi \left( t_0 \right)$$
при $t \to t_0$.

Сходимость по вероятности сохраняется при непрерывных операциях.
Сумма --- непрерывная операция.

Знаем, что $f$ --- непрерывна, то есть если $t \to t_0$,
то $f \left( t \right) \to f \left( t_0 \right) $.
От $ \omega $ тут зависимости нет.
Эту сходимость можно интерпретировать как сходимость почти наверное, следовательно,
$$f \left( t \right) \overset{P}{ \to }
  f \left( t_0 \right).$$

Значит и сумма будет сходиться.
Значит, отсюда следует, что $ \eta $ --- стохастически непрерывен.

Теперь предоложим, что вся сумма стохастически непрерывна.

Пусть $ \eta $ --- стохастически непрерывен в $t_0$.
Это значит, что
$$ \eta \left( t \right) \overset{P}{ \to } \eta \left( t_0 \right), \,
  t \to t_0.$$

Для $ \eta $ и $ \xi $ мы знаем, что есть сходимость по вероятности.
Надо взять разность.
Разность --- это $f$, то есть
$$ \begin{cases}
    \xi \left( t \right) + f \left( t \right) \overset{P}{ \to }
    \xi \left( t_0 \right) + f \left( t_0 \right), \\
    \xi \left( t \right) \overset{P}{ \to } \xi \left( t_0 \right).
  \end{cases}$$

Вычтем из первого второе
$$f \left( t \right) \overset{P}{ \to }
  f \left( t_0 \right).$$
Нужно проверить, что для неслучайной функции сходимость по вероятности и просто сходимость ---
одно и то же.
Сходимость по вероятности:
$ \forall \varepsilon > 0 \qquad
  P \left\{ \left| f \left( t \right) - f \left( t_0 \right) \right| > \varepsilon \right\} \to 0,
  \, t \to t_0$.
Это есть.
Просто сходимость: $ \forall \varepsilon > 0 \, \exists \delta > 0$,
чтобы выполнялось соотношение
$ \left| f \left( t \right) - f \left( t_0 \right) \right| <
  \varepsilon $
при $ \left| t - t_0 \right| < \delta $.
Это нужно проверить.

$ \forall \varepsilon > 0 \, \forall \alpha > 0 \, \exists \delta > 0 \qquad
  P \left\{ \left| f \left( t \right) - f \left( t_0 \right) \right| > \varepsilon \right\} <
  \alpha $
при $ \left| t - t_0 \right| < \delta $.

Функция $f$ --- неслучайная функция, то есть событие неслучайно.
Его вероятность равна или нулю, или единице.
Если $ \alpha > 1$, то вероятность равна нулю.
Значит, $ \left| f \left( t \right) - f \left( t_0 \right) \right| < \varepsilon $ при
$ \left| t - t_0 \right| <
  \delta $.
То есть $ \alpha < 1$.

Тогда
$P \left\{ \left| f \left( t \right) - f \left( t_0 \right) \right| > \varepsilon \right\} =
  0$.
Из этого следует, что при $ \left| t - t_0 \right| < \delta $ выполняется дополнение
$ \left| f \left( t \right) - f \left( t_0 \right) \right| \leq
  \varepsilon $.
Это и значит непрерывность в точке $t_0$.
Таким образом, для неслучайных величин все сходимости равносильны.

\subsubsection*{6.9}

\textit{Задание.}
Пусть $ \left\{ X \left( t \right), \, t \in T \right\} $ --- случайный процесс такой, что
$$MX \left( t \right) = 0, \,
  MX^2 \left( t \right) = 1$$
для произвольного $t \in T$.
\begin{enumerate}[label=\alph*)]
  \item Докажите, что $ \left| MX \left( t \right) X \left( t + h \right) \right| \leq 1$
  для произвольного $h > 0$ и произвольного $t \in \left[ 0, T - h \right] $.
  \item Допустим, что для некоторых $ \lambda < \infty, \, p > 1$ и $h_0 > 0$
  $$M \left[ X \left( t \right) X \left( t + h \right) \right] \geq
    1 - \lambda h^p$$
  для произвольного $h \in \left( 0, h_0 \right] $.
  Докажите, что $ \left\{ X \left( t \right), \, t \in T \right\} $ имеет непрерывную модификацию.
\end{enumerate}

\textit{Решение.}
\begin{enumerate}[label=\alph*)]
  \item Пусть $X \left( t \right) = \xi $ и $X \left( t + h \right) = \eta $.
  Тогда
  $$ \left| M \xi \eta \right| \leq
    M \left| \xi \eta \right| \leq
    \left( M \left| \xi \right|^p \right)^{ \frac{1}{p}}
    \left( M \left| \eta \right|^q \right)^{ \frac{1}{q}},$$
  где
  $$ \frac{1}{p} + \frac{1}{q} =
    1$$
  (неравенство Гёльдера).
  Возьмём $p = q = 2$.

  Получаем неравенство Коши-Буняковского
  $$M \left[ X \left( t \right) X \left( t + h \right) \right] \leq
    \left\{ M \left| X \left( t \right) \right|^2 \right\}^{ \frac{1}{2}}
    \left\{ M \left| X \left( t + h \right) \right|^2 \right\}^{ \frac{1}{2}} =
    1 \cdot 1 =
    1.$$
  \item Будем пользоваться достаточным условием Колмогорова
  $$ \exists \alpha, \beta, C > 0 \, : \,
    \forall t_1, t_2 \in \left[ 0, 1 \right] \qquad
    M \left| \xi \left( t_1 \right) - \xi \left( t_2 \right) \right|^{ \alpha } \leq
    C \left| t_1 - t_2 \right|^{1 - \beta }.$$
  Тогда у процесса будут непрерывные модификации.
  Нужно оценить
  $$M \left| X \left( t + h \right) - X \left( t \right) \right|^2 =
    M\left[X^2\left(t+h\right)-2X\left(t+h\right)X\left(t\right)+X^2\left(t\right)\right] =$$
  Воспользуемся линейностью математического ожидания
  $$= MX^2 \left( t + h \right) - 2M \left[  X \left( t + h \right) X \left( t \right) \right] +
    MX^2 \left( t \right).$$
  Здесь первое и последнее слагаемые равны единице, а второе не меньше $1 - \lambda h^p$.
  Тогда
  $$MX^2 \left( t + h \right) - 2M \left[  X \left( t + h \right) X \left( t \right) \right] +
    MX^2 \left( t \right) \geq
    2 - 2 \left( 1 - \lambda h^p \right) =
    2 \lambda h^p,$$
  где $2 \lambda = const, \, p = 1 + \beta $.

  Теорема Колмогорова работает с $ \alpha = 2, \, C = 2 \lambda $ и $ \beta = p - 1$.

  Значит, такой процесс имеет непрерывные модификации.
\end{enumerate}
