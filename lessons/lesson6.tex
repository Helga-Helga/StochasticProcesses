\addcontentsline{toc}{chapter}{Занятие 6. Стохастическая непрерывность случайного процесса.
                                Существование непрерывной модификации}
\chapter*{Занятие 6. Стохастическая непрерывность случайного процесса.
          Существование непрерывной модификации}

\addcontentsline{toc}{section}{Контрольные вопросы и задания}
\section*{Контрольные вопросы и задания}

\subsubsection*{Приведите опредедение стохастически непрерывного процесса.}

Стохастически непрерывный процесс:
$ \xi \left( t \right) \overset{P} \xi \left( t_0 \right), \, t \to t_0$.

Это означает, что
$ \forall \varepsilon > 0 \qquad
  P \left( \left| \xi \left( t \right) - \xi \left( t_0 \right) \right| > \varepsilon \right) \to 0,
  \, t \to t_0$.

\subsubsection*{Сформулируйте достаточное условие существования непрерывной модификации случайного
                процесса.}

Пусть $ \xi \left( t \right), \, t \in \left[ 0, 1 \right] $ удовлетворяет условию
$$ \exists \alpha, \beta, C > 0: \qquad
  \forall t_1, t_2 \in \left[ 0, 1 \right] \qquad
  M \left| \xi \left( t_1 \right) =
  \xi \left( t_2 \right) \right|^{ \alpha } \leq C \left| t_1 - t_2 \right|^{1 + \beta }.$$
Тогда $ \xi $ имеет непрерывную модификацию.
